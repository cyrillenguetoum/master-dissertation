\chapter*{Conclusion et Perspectives}
\addcontentsline{toc}{chapter}{Conclusion et perspectives}

La recherche de parasites gastro-intestinaux de bovins dans l’arrondissement de Bangangté nous a permis d’identifier 10 espèces d’helminthes : \textit{Haemonchus contortus}, \textit{Ostertagia ostertagi}, \textit{Cooperia} spp., \textit{Trichostrongylus axei}, \textit{Strongyloides papillosus}, \textit{Nematodirus battus}, \textit{Moniezia benedeni}, \textit{Trichuris} spp., \textit{Fasciola gigantica} et \textit{Paramphistomum cervi} avec pour prévalences respectives : 44,67\% ; 7,67\% ; 6\% ; 13,33\% ; 4\% ; 8,33\% ; 1,33\% ; 0,33\% ; 38,33\% et 3,67\%. Les parasites \textit{O. ostertagi} et \textit{Cooperia} spp. sont associés à une baisse de l’hématocrite tandis que la baisse de la NEC est associée aux espèces \textit{O. ostertagi} et \textit{N. battus}. Les combinaisons de ces parasites associées à une baisse significative de l’hématocrite sont \textit{H. contortus} avec \textit{Cooperia} spp et \textit{O. ostertagi} avec \textit{S. papillosus} tandis que \textit{H. contortus} - \textit{N. battus} et \textit{H. contortus} - \textit{F. gigantica} - \textit{T. axei} sont associés à une baisse de la NEC. Afin de poursuivre l'exploration de la thématique abordée par ce travail, nous suggérons :
\begin{itemize}
\item Une comparaison des techniques de management des fermes et des environnements agroécologiques afin de mieux expliquer la variation des prévalences de parasites ;
\item L’extraction, la purification et l’administration des œufs de diverses combinaisons d’espèces parasitaires à des animaux expérimentaux, afin d’analyser davantage la morbidité due aux co-infestations ;
\item Une cartographie des aires de pâturage des exploitations bovines dans le but d’étudier la distribution spatiale des espèces d’helminthes et donc la constitution des communautés de parasites.
\end{itemize}