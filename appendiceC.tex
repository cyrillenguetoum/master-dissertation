%\chapter*{Annexe C: Détermination de l'âge des bovins} \label{title:guide-age}
%\addcontentsline{toc}{chapter}{Annexe C: Détermination de l'âge des bovins}
\chapter{Détermination de l'âge des bovins} \label{annex:age-guide}

\renewcommand{\thefigure}{C.\arabic{figure}}
\setcounter{figure}{0}
\renewcommand{\thetable}{C.\Roman{table}}
\setcounter{table}{0}


Il existe trois méthodes de détermination de l'âge des zébus soudaniens: l'observation de la dentition, le décompte des anneaux sur les cornes et la mesure de la toupe de la queue. De ces trois méthodes, celle reposant sur l'observation de la dentition est la plus précise. Trois critères président à la détermination de l'âge par la dentition selon \shortcite{RN64, RN65}:
\begin{itemize}
	\item L'usure des dents caduques ou dents de lait;
	\item L'observation de l'apparition des incisives de remplacement;
	\item L'observation de l'usure, du nivellement et de l'écartement des incisives de remplacement.
\end{itemize} 

\begin{longtable}{lC{7cm}C{4cm}}
\caption[Détermination de l'âge des bovins par l'examen de la dentition]{Détermination de l'âge des bovins par l'examen de la dentition selon \shortciteA{RN65}}\\
\toprule
\textbf{Age} & \textbf{Dentition} & \textbf{Schema} \\
\midrule
\endfirsthead
\caption*{Méthode de détermination de l'âge des bovins par la dentition selon \shortciteA{RN65} (\textit{suite})} \\
\toprule
\textbf{Age} & \textbf{Dentition} & Schema \\
\midrule
\endhead
\textbf{< 1 mois} & Au moins deux incisives temporaires sont présents (pinces). Dans le mois l'ensemble des 8 incisives temporaires apparaissent.  & \includegraphics[width=\linewidth] {images/dentition/1mois.png}\\
\midrule
\textbf{2 ans} & Les deux premières incisives temporaires sont remplacées par les permanentes pendant la première année de vie. Les premières incisives permanentes ou pinces atteignent leur plein developpement. & \includegraphics[width=\linewidth] {images/dentition/2ans.png}\\
\midrule
\textbf{2 ans et demi} & Les premières mitoyennes permanentes se developpent (de part et d'autre de la paire centrale d'incisives permanente). & \includegraphics[width=\linewidth] {images/dentition/2ansetdemi.png}\\
\midrule
\textbf{3 ans et demi} & Les deuxièmes mitoyennes permanentes se developpent et atteignent la taille des premières. A 4 ans elles commencent à s'user. & \includegraphics[width=\linewidth] {images/dentition/3ansetdemi}\\
\midrule
\textbf{4 ans et demi} & Les coins sont remplacées. A 5 ans les incisives sont totalement développées avec les coins. & \includegraphics[width=\linewidth] {images/dentition/4ansetdemi}\\
\midrule
\textbf{5 à 6 ans} & Les pinces permanentes sont nivelées. Le mitoyennes le sont partiellement et les coins montrent des signes d'usure. & \includegraphics[width=\linewidth] {images/dentition/5a6ans}\\
\midrule
\textbf{7 à 10 ans} & à 7-8 ans les pinces montrent des signes d'usure prononcés, à 8-9 ans les mitoyennes montrent à leur tour des signes d'usure prononcés et vers 10 ans les coins s'usent également. & \includegraphics[width=\linewidth] {images/dentition/7a10ans}\\
\midrule
\textbf{12 ans} & à partir de 6 ans l'arc buccal perd progressivement sa forme arrondie, pour devenir totalement plat vers 12 ans. De même les dents adoptent une forme triangulaire et sont séparées les unes des autres par des espaces vides. Ces transformations s'accentuent avec l'âge. & \includegraphics[width=\linewidth] {images/dentition/12ans} \\

\bottomrule
\end{longtable}