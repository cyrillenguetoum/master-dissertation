%\chapter*{Annexe D: Choix des combinaisons d'espèces à inclure dans les modèles statistiques}
%\addcontentsline{toc}{chapter}{Annexe D: Choix des combinaisons de parasites à inclure dans les modèles statistiques}
\chapter{Choix des combinaisons d'espèces à inclure dans les modèles statistiques} \label{annex:statistics-guide}

\renewcommand{\thefigure}{D.\arabic{figure}}
\setcounter{figure}{0}
\renewcommand{\thetable}{D.\Roman{table}}
\setcounter{table}{0}

\par Selon l’ACP ou PCA (\textit{Principal Component Analysis} ou Analyse de composantes principales) dont les résultats sont présentés par les figures \ref{fig:res.pca.indivis.assoc-1} et \ref{fig:res.pca.indivis.assoc-2}, les espèces \textit{Cooperia} spp., \textit{Ostertagia ostertagi }, \textit{Strongyloides papillosus}, \textit{Haemonchus contortus}, \textit{Nematodirus battus} et \textit{Trichostrongylus axei} sont négativement corrélés à l’hématocrite, si l'on se réfère à leur projection sur la composante principale 1. De même les espèces \textit{Haemonchus contortus}, \textit{Nematodirus battus}, \textit{Trichostrongylus axei} et \textit{Fasciola gigantica} sont négativement corrélés à la NEC si l'on se réfère à leurs projections sur la composante principale numéro 2. 

\begin{figure}[!ht]
	{
		\centering 
		\subfloat[Effets de la présence des espèces parasitaires\label{fig:res.pca.indivis.assoc-1}]
		{\includegraphics[width=0.5\linewidth,]{images/r_plots/pca_parasites_individuels_2.png} }
		\subfloat[Effets des OPGs\label{fig:res.pca.indivis.assoc-2}]
		{\includegraphics[width=0.5\linewidth,]{images/r_plots/pca_parasites_individuels_1.png} }
	}
	\caption{Représentation de l'analyse des composantes principales entre les variables de présence des parasites, les variables de décompte des \oe ufs avec les variables hématocrite et NEC}\label{fig:res.pca.indivis.assoc}
\end{figure}

\par Donc les combinaisons d'espèces dont on a recherché les effets sur les paramètres sanitaires sont constitués à partir de deux groupes:
\begin{itemize}
	\item Groupe ayant un effet sur l’hématocrite: \textit{Ostertagia ostertagi}, \textit{Cooperia} spp., \textit{Strongyloides papillosus}, \textit{Trichostrongylus axei}, \textit{Nematodirus battus} et \textit{Haemonchus contortus};
	\item Groupe ayant un effet sur la NEC: \textit{Fasciola gigantica}, \textit{Nematodirus battus}, \textit{Haemonchus contortus} et \textit{Trichostrongylus axei}.
\end{itemize}

Les modèles statistiques ont permi d'établir l'association entre les paramètres sanitaires 
(variables réponses) et les combinaisons de parasites. Les combinaisons de
d'espèces parasitaires devant faire partie des modèles statistiques ne doivent pas exprimer 
de la multicollinéarité ou l'exprimer à un faible degré. La multicollinéarité est la 
corrélation forte qui peut exister entrer des variables explicatives d'un modèle statistique,
et ce phénomène nuit à la précision du modèle. En effet la variables fortement corrélées 
entre elles représentent la même information. Il est donc nécessaire de n'inclure dans les 
modèles qu'une seule variable des couples fortement corrélés \shortcite{RN67}.


\begin{figure}[!ht]
	\includegraphics[width=0.8\linewidth,]{images/r_plots/corr_verification-2}
	\caption{\label{fig:matrice_corr_nec}Matrice de corrélation entre les variables 
	explicatives du modèle de regression logistique ordinale de la NEC}
\end{figure}

\par La colinéarité entre des variables se vérifie par le calcul des 
coefficients de corrélation (de Pearson) entre elles.
Les variables explicatives de la NEC sont représentés dans la matrice de corrélation 
ci-dessous (figure \ref{fig:matrice_corr_nec}). Il ressort de cette figure que les 
couples de variables: combi\_47 - combi\_50 (r = 0,8), combi\_48 - combi\_51 
(r = 0,8) et combi\_49 - combi\_52 (r = 0,7) se distinguent par les corrélations 
fortes entre leurs membres. Il n'est donc pas nécessaire de tous les inclure dans 
le modèle. On pourrait choisir au sein de chaque couple une seule variable devant 
faire partie du modèle. Au final nous avons choisi les variables combi\_47, 
combi\_48 et combi\_52. Combi\_52 a été préféré à combi\_49 parcequ'il réduit 
davantage l'AIC (\textit{Akiake Information Criterion}) du modèle. 

\begin{figure}[!ht]
	\includegraphics[width=0.8\linewidth,]{images/r_plots/corr_verification-1}
	\caption{\label{fig:matrice_corr_hem}Matrice de corrélation entre les variables 
	explicatives du modèle de regression linéaire de l'hematocrite via les 
	combinaisons de parasites}
\end{figure}

Ci-dessus on peut voir une représentation graphique de la matrice de corrélation 
entre les variables explicatives de l'hématocrite. Les variables combi\_16 à 
combi\_43 présentent des corrélations fortes avec d'autres variables et les 
représenter ici aurait rendu le graphe illisible. Nous avons donc choisi de les 
retirer. Les variables combi\_5 et combi\_12 ont été rétirés du modèle à cause des 
corrélations fortes qu'elles forment avec d'autres variables.

Les associations spécifiques retenues sont fournies dans le tableau \ref{tab:combi_building_haem} pour les combinaisons pouvant modifier l’hématocrite et le tableau \ref{tab:combi_building_nec} pour combinaisons pouvant modifier la NEC.

\begin{table}[!ht]
	\caption{\label{tab:combinaisons_parasites}Arrangements de parasites ayant potentiellement un effet sur les paramètres de morbidité et effectivement apparus dans les données}
	\begin{subtable}[t]{0.48\textwidth}
		\centering
		\begin{tabular}[t]{ll}
			\toprule
			\textbf{Variables} & \textbf{Combinaisons}\\
			\midrule
			combi\_1 & Oo \& C\_spp\\
			combi\_2 & Oo \& Sp\\
			combi\_3 & Oo \& Hc\\
			combi\_4 & Oo \& Nb\\
			combi\_6 & C\_spp \& Sp\\
			combi\_7 & C\_spp \& Hc\\
			combi\_8 & C\_spp \& Nb\\
			combi\_9 & C\_spp \& Ta\\
			combi\_10 & Sp \& Hc\\
			combi\_11 & Sp \& Nb\\
			combi\_13 & Hc \& Nb\\
			combi\_14 & Hc \& Ta\\
			combi\_15 & Nb \& Ta\\			
			\bottomrule
		\end{tabular}
		\caption{\label{tab:combi_building_haem}sur l'hématocrite}
	\end{subtable}
	\begin{subtable}[t]{0.48\textwidth}
		\centering
		\begin{tabular}[t]{ll}
			\toprule
			\textbf{Variables} & \textbf{Combinaisons}\\
			\midrule
			combi\_44 & Hc \& Nb \\
			combi\_45 & Hc \& Fg \\
			combi\_46 & Hc \& Ta \\
			combi\_47 & Nb \& Fg \\
			combi\_48 & Nb \& Ta \\
			combi\_52 & Hc \& Fg \& Ta \\
			\bottomrule
		\end{tabular}
		\caption{\label{tab:combi_building_nec}sur la NEC}
	\end{subtable}
\end{table}

\textbf{\underline{Légende}}: \textbf{Oo}: \textit{Ostertagia ostertagi}; \textbf{Hc}: \textit{Haemonchus contortus};
\textbf{Nb}: \textit{Nematodirus battus}; \textbf{Sp}: \textit{Strongyloides papillosus}; \textbf{C$\_$spp}: \textit{Cooperia} spp.; \textbf{T$\_$spp}: \textit{Trichuris} spp.; \textbf{Ta}: \textit{Trichostrongylus axei}; \textbf{Fg}: \textit{Fasciola gigantica}; \textbf{Pc}: \textit{Paramphistomum cervi}; \textbf{Mb}: \textit{Moniezia benedeni}.
