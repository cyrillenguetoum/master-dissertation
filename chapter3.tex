\chapter{Résultats et Discussion}
%\chapter*{Chapitre 3: Résultats et Discussion}
%\addcontentsline{toc}{chapter}{Chapitre 3: Résultats et Discussion}
%\setcounter{section}{0} % reinitialise le compteur des sections
%\renewcommand*{\theHsection}{chX.\the\value{section}}

\section{Résultats}
\subsection{Caractéristiques des espèces parasites gastro-intestinaux reçensés}
\subsubsection{Description des espèces parasites reçensés}
Dix espèces de parasites gastro-intestinaux ont été détectés. Il s’agit de : \textit{Haemonchus
contortus}, \textit{Ostertagia ostertagi}, \textit{Cooperia} spp., \textit{Trichostrongylus axei}, \textit{Strongyloides papillosus}, \textit{Nematodirus battus}, \textit{Moniezia benedeni}, \textit{Trichuris} spp., \textit{Fasciola gigantica} et \textit{Paramphistomum cervi}.

\begin{longtable}{lcC{5cm}C{4cm}}
	\caption{Description morphométrique des oeufs des espèces parasites reçensées}\\
	\toprule
	\textbf{Espèces parasites} &  & \textbf{Description}  & \textbf{Photographie} \\
	\midrule
	\endfirsthead
	\toprule
	\textbf{Espèces parasites} &  & \textbf{Description}  & \textbf{Photographie} \\
	\midrule
	\endhead
	\textit{Haemonchus contortus} & Nematode & \OE uf de forme ovoïde, possédant une vingtaine de blastomères et mesurant en moyenne 70$\pm$10 x 45$\pm$5$\mu$m. & \includegraphics[width=\linewidth] {images/tofs_microscope/haemonchus_contortus.png} \\
	\midrule
	\textit{Ostertagia ostertagi} & Nematode & \OE uf ovoïde et mesurant en moyenne 80-85 x 40-45$\mu$m. Les blastomères sont indiscernables du fait de leur nombre très élevé et de leurs tailles très réduite. & \includegraphics[width=\linewidth]
	{images/tofs_microscope/ostertagia_ostertagi.png} \\
	\midrule
	\textit{Cooperia} spp. & Nematode & \OE ufs à paroi mince, de forme allongée et aux blastomères de couleur brune. & \includegraphics[width=\linewidth]
	{images/tofs_microscope/cooperia_spp.png} \\
	\midrule
	\textit{Trichostrongylus axei} & Nematode & \OE uf à paroi très mince, de forme allongée, à bouts arrondis et de dimensions 79-92 x 31-41$\mu$m. Les blastomères sont concentrés au centre de l'\oe uf et n'emplissent pas les apex de l'\oe uf. & \includegraphics[width=\linewidth]
	{images/tofs_microscope/trichostrongylus_axei.png} \\
	\midrule 
	\textit{Strongyloides papillosus} & Nematode & \OE uf de couleur grise, de forme ovoïde, mesurant en moyenne 40-60 x 20-25$\mu$m et contenant un embryon au moment de son expulsion dans les matières fécales. & \includegraphics[width=\linewidth]
	{images/tofs_microscope/strongyloides_papillosus.png} \\
	\midrule
	\textit{Nematodirus battus} & Nematode & \OE uf de grande taille (175-260 x 106-
	110$\mu$m en moyenne), de forme ovale allongée, et contenant en moyenne 8 blastomères. & \includegraphics[width=\linewidth]
	{images/tofs_microscope/nematodirus_battus.png} \\
	\midrule
	\textit{Moniezia benedeni} & Cestode & Oncomère translucide, de forme quadrilatère, ronde ou triangulaire, mesurant environ 30$\mu$m et entouré d'une membrane protectrice. & \includegraphics[width=\linewidth]{images/tofs_microscope/moniezia_benedeni.png} \\
	\midrule
	\textit{Trichuris} spp. & Nematode & \OE uf ayant la forme d'un ballon de rugby, mesurant 60-73 x 25-35$\mu$m, avec des extrémités se terminant par des ouvertures recouvertes de bouchons bien visibles. & \includegraphics[width=\linewidth]
	{images/tofs_microscope/trichuris_spp.png} \\
	\midrule
	\textit{Fasciola gigantica} & Trematode & \OE uf de grande taille de couleur jaune, mesurant 130-150 x 60-90$\mu$m, de forme ovale allongée et presentant un contenu à l'aspect granuleux. Un opercule est quelquefois visible à l'une des extrémités. & \includegraphics[width=\linewidth]
	{images/tofs_microscope/fasciola_gigantica.png} \\
	\midrule 
	\textit{Paramphistomum cervi} & Trematode & \OE uf de grande taille ressemblant fortement à ceux de \textit{Fasciola} spp. (de dimensions 130-150 x 60-90$\mu$m), mais de couleur grise. Une ligne fine délimite l'opercule. & \includegraphics[width=\linewidth]
	{images/tofs_microscope/paramphistomum_cervi.png} \\
	\bottomrule
\end{longtable}


\subsubsection{Prévalence des helminthes parasites gastro-intestinaux}

\textit{Haemonchus contortus} est l'espèce ayant la présence la plus forte, suivi respectivement de \textit{Fasciola gigantica}, \textit{Trichostrongylus axei}, \textit{Nematodirus battus}, \textit{Ostertagia ostertagi}, \textit{Cooperia} spp., \textit{Strongyloides papillosus}, \textit{Paramphistomum cervi}, \textit{Moniezia benedeni} et \textit{Trichuris} spp. Les prévalences et les nombres d'\oe ufs par gramme de fèces (OPG) pour chacune des espèces reçensées sont résumées dans le tableau \ref{tab:prevalence_frequentiste}. De toutes les espèces reçensées, \textit{Haemonchus contortus} a l'OPG moyen le plus élevé. Elle était suivie de \textit{Moniezia benedeni} qui malgré sa faible prévalence, présente des OPGs allant jusqu'à 10650 oncomères par gramme de fèces  pour un seul animal.

\begin{table}[!h]
	
	\caption{\label{tab:prevalence_frequentiste}Prévalences et OPG moyens des espèces parasites reçensées}
	\centering
	\begin{tabular}[t]{lcc}
		\toprule
		\textbf{Espèces} & \textbf{Prévalence en \% ± SE} & \textbf{OPG moyens ± SD}\\
		\midrule
		\cellcolor{gray!6}{\textit{Haemonchus contortus}} & \cellcolor{gray!6}{44,67 ± 5,63} & \cellcolor{gray!6}{103,33 ± 230,00}\\
		\textit{Ostertagia ostertagi} & 7,67 ± 3,01 & 8,50 ± 40,10\\
		\cellcolor{gray!6}{\textit{Cooperia }spp.} & \cellcolor{gray!6}{6,00 ± 2,69} & \cellcolor{gray!6}{16,50 ± 170,39}\\
		\textit{Trichostrongylus axei }& 13,33 ± 3,85 & 13,83 ± 46,45\\
		\cellcolor{gray!6}{\textit{Strongyloides papillosus}} & \cellcolor{gray!6}{4,00 ± 2,22} & \cellcolor{gray!6}{2,33 ± 12,04}\\
		\textit{Nematodirus battus} & 8,33 ± 3,13 & 8,50 ± 36,38\\
		\cellcolor{gray!6}{\textit{Moniezia benedeni}} & \cellcolor{gray!6}{1,33 ± 1,30} & \cellcolor{gray!6}{54,83 ± 680,49}\\
		\textit{Trichuris} spp. & 0.33 ± 0.65 & 0.17 ± 2.89\\
		\cellcolor{gray!6}{\textit{Paramphistomum cervi}} & \cellcolor{gray!6}{3,67 ± 2,13} & \cellcolor{gray!6}{-}\\
		\textit{Fasciola gigantica} & 38,33 ± 5,50 & -\\
		\midrule
		\multicolumn{3}{l}{SE: erreur standard; SD: écart type}\\
		\bottomrule
	\end{tabular}
\end{table}

\par Nous avons déterminé que 29,3\% des animaux ne présentaient aucun parasite contre 70,6\% des
animaux qui présentaient au moins un parasite. Les richesses spécifiques des helminthes ont été classifiées en 5 niveaux: aucune espèce, une espèce, deux espèces, trois espèces et quatre espèces ou plus. Les taux d'animaux correspondant à chaque niveaux de richesse parasitaire ont été évalués: 42\% des animaux présentaient au moins deux espèces parasites, 10\% au moins trois espèces parasites et 3\% au moins 4 espèces parasites (tableau  \ref{tab:prevalences_coinfestations}).

\begin{table}[!h]
	
	\caption{\label{tab:prevalences_coinfestations}Taux d'animaux correspondant aux niveaux de richesse parasitaire}
	\centering
	\begin{tabular}[t]{lc}
		\toprule
		\textbf{Associations} & \textbf{Prévalence en \% ± SE}\\
		\midrule
		\cellcolor{gray!6}{aucune espèce parasite} & \cellcolor{gray!6}{29,33 ± 5,15}\\
		au moins une espèce parasite & 70,67 ± 5,15\\
		\cellcolor{gray!6}{au moins 2 espèces parasites} & \cellcolor{gray!6}{42,00 ± 5,59}\\
		au moins 3 espèces parasites & 10,00 ± 3,39\\
		\cellcolor{gray!6}{au moins 4 espèces parasites} & \cellcolor{gray!6}{3,00 ± 1,93}\\
		\midrule
		\multicolumn{2}{l}{SE: erreur standard}\\
		\bottomrule
	\end{tabular}
\end{table}


\subsubsection{Facteurs de variation de la richesse parasitaire}
Les taux d'animaux correspondant aux richesses spécifiques des helminthes sont classifiés en fonction de quelques facteurs biotiques ou abiotiques dans le tableau \ref{tab:proportions_of_coinfestations_by_factors}. Les cas de richesse spécifiques lourdes (quatre espèces et plus) sont globalement moins fréquents que les autres cas de figure. La majorité des sujets porte un ou deux parasites (60,67\%). 
\par Les catégories d'âges représentent les principaux facteurs de variation de la richesse parasitaire avec des différences statistiquement significatives entre les proportions de veaux, taurillons et adultes parasités.
\par Il n'y a pas eu de différence significative entre les proportions d'animaux touchés par le monoparasitisme quel que soit le facteur étudié. Pour tous les niveaux de richesse parasitaire, la race des animaux et la présence de tiques n'influencent pas les prévalences. Les males sont plus souvent indemnes de parasites que les femelles, avec une différence significative (p = 0,03). Les veaux et les velles sont sujets à une richesse spécifique plus accentuée que les animaux plus âgés.

\blandscape

\begin{longtable}{lcccccc}
	
	\caption{\label{tab:proportions_of_coinfestations_by_factors}Proportions d'animaux par richesse spécifique en fonction de facteurs biotiques et abiotiques} \\
	\toprule
	
	Facteurs & \textbf{Total}, N = 300\textsuperscript{1} & \textbf{0}, N = 88\textsuperscript{1} & \textbf{1}, N = 86\textsuperscript{1} & \textbf{2}, N = 95\textsuperscript{1} & \textbf{3}, N = 22\textsuperscript{1} & \textbf{4}, N = 9\textsuperscript{1} \\ 
	\midrule
	\textbf{Sexe} &  & * & NS & NS & NS & NS \\  
	Femelle & 222 (74,00\%) & 57 (25,67\%) & 70 (31,53\%) & 75 (33,78\%) & 14 (6,30\%) & 6 (2,70\%)  \\ 
	Mâle & 78 (26,00\%) & 31 (39,74\%) & 16 (20,51\%) & 20 (25,64\%) & 8 (10,26\%) & 3 (3,85\%)  \\
	\midrule
	\textbf{Race} &  & NS & NS & NS & NS & NS \\ 
	Bokolo & 3 (1,00\%) & 0 (0,00\%) & 1 (33,33\%) & 2 (66,66\%) & 0 (0,00\%) & 0 (0,00\%)  \\ 
	Goudali & 202 (67,33\%) & 61 (30,20\%) & 59 (29,20\%) & 57 (28,22\%) & 16 (7,92\%) & 9 (4,45\%)  \\ 
	Red-Fulani & 36 (12,00\%) & 10 (27,77\%) & 12 (33,33\%) & 11 (30,55\%) & 3 (8,33\%) & 0 (0,00\%)  \\ 
	White-Fulani & 59 (19,67\%) & 17 (28,81\%) & 14 (23,73\%) & 25 (42,37\%) & 3 (5,08\%) & 0 (0,00\%)  \\ 
	\midrule
	\textbf{Ages} &  & * & NS & * & NS & * \\ 
	adulte & 173 (57,67\%) & 42 (24,28\%) & 56 (32,37\%) & 62 (35,84\%) & 11 (6,36\%) & 2 (1,16\%)  \\ 
	taurillon\_genisse & 38 (12,67\%) & 11 (28,95\%) & 10 (26,32\%) & 15 (39,47\%) & 2 (5,26\%) & 0 (0,00\%)  \\ 
	veau\_velle & 89 (29,67\%) & 35 (39,33\%) & 20 (22,47\%) & 18 (20,22\%) & 9 (10,11\%) & 7 (7,87\%)  \\ 
	\midrule
	\textbf{Tiques} & & NS & NS & NS & NS & NS \\ 
	Non & 13 (4,33\%) & 6 (46,15\%) & 2 (15,38\%) & 3 (23,08\%) & 1 (7,69\%) & 1 (7,69\%)  \\ 
	Oui & 287 (95,67\%) & 82 (28,57\%) & 84 (26,27\%) & 92 (32,06\%) & 21 (7,32\%) & 8 (2,79\%) \\
	\midrule
	\multicolumn{7}{l}{\textsuperscript{1}n (\%)} \\
	\multicolumn{7}{l}{NS: P-value Non-significative (supérieure à 0,05)} \\
	\multicolumn{7}{l}{Codes de significance: 0 < *** < 0,001 < ** < 0,01 < * < 0,05 < NS } \\
	\multicolumn{7}{l}{P-values obtenues après réalisation du test du $\chi$\textsuperscript{2} d'homogenéité des proportions} \\
	\bottomrule
\end{longtable}

\elandscape

\subsection{Effets sanitaires des infestations}
\subsubsection{Effets de la présence des espèces reçensés sur les paramètres sanitaires}

La présence de certains parasites entraine une modification significative de l’hématocrite et de la NEC comme le montre le tableau \ref{tab:effets_presence_parasites}.

\begin{longtable}[t]{p{2.5cm}c>{\centering}p{4cm}c>{\centering}p{3cm}c}
	
	\caption{\label{tab:effets_presence_parasites}Hematocrites et NECs moyens des animaux en fonction de la présence ou non des espèces parasitaires reçensés}\\
		\toprule
		\textbf{Espèces} & \textbf{Présence} & \textbf{Hematocrites moyens ± ET (min; max)} & \textbf{p-values} & \textbf{NEC moyennes ± ET (min; max)} & \textbf{p-values}\\
		\midrule
		& non & 30.27 ± 4.39 \linebreak( 17.03 ; 40.15 ) &  & 2.55 ± 0.32 \linebreak( 2 ; 3.5 ) & \\
		\cmidrule{2-3}
		\cmidrule{5-5}
		\multirow{-2}{2.5cm}{\textit{Haemonchus contortus}} & oui & 29.96 ± 4.90 \linebreak( 19.09 ; 38.64 ) & \multirow{-2}{*}{\raggedleft\arraybackslash 0.56} & 2.55 ± 0.31 \linebreak( 2 ; 3 ) & \multirow{-2}{*}{\raggedleft\arraybackslash 0.87}\\
		\cmidrule{1-6}
		& non & 30.29 ± 4.54 \linebreak( 17.03 ; 40.15 ) &  & 2.56 ± 0.32 \linebreak( 2 ; 3.5 ) & \\
		\cmidrule{2-3}
		\cmidrule{5-5}
		\multirow{-2}{2.5cm}{\textit{Ostertagia ostertagi}} & oui & 28.23 ± 5.15 \linebreak( 19.09 ; 37 ) & \multirow{-2}{*}{\raggedleft\arraybackslash 0.06} & 2.41 ± 0.33 \linebreak( 2 ; 3 ) & \multirow{-2}{*}{\raggedleft\arraybackslash 0.03*}\\
		\cmidrule{1-6}
		& non & 30.30 ± 4.51 \linebreak( 17.03 ; 40.15 ) &  & 2.56 ± 0.32 \linebreak( 2 ; 3.5 ) & \\
		\cmidrule{2-3}
		\cmidrule{5-5}
		\multirow{-2}{2.5cm}{\textit{Cooperia} spp.} & oui & 27.40 ± 5.44 \linebreak( 19.09 ; 36.88 ) & \multirow{-2}{*}{\raggedleft\arraybackslash 0.02*} & 2.44 ± 0.29 \linebreak( 2 ; 3 ) & \multirow{-2}{*}{\raggedleft\arraybackslash 0.14}\\
		\cmidrule{1-6}
		& non & 30.28 ± 4.57 \linebreak( 17.03 ; 40.15 ) &  & 2.55 ± 0.31 \linebreak( 2 ; 3.5 ) & \\
		\cmidrule{2-3}
		\cmidrule{5-5}
		\multirow{-2}{2.5cm}{\textit{Trichostrongylus axei}} & oui & 29.14 ± 4.88 \linebreak( 19.09 ; 38.64 ) & \multirow{-2}{*}{\raggedleft\arraybackslash 0.11} & 2.54 ± 0.35 \linebreak( 2 ; 3 ) & \multirow{-2}{*}{\raggedleft\arraybackslash 0.84}\\
		\cmidrule{1-6}
		& non & 30.25 ± 4.52 \linebreak( 17.03 ; 40.15 ) &  & 2.55 ± 0.32 \linebreak( 2 ; 3.5 ) & \\
		\cmidrule{2-3}
		\cmidrule{5-5}
		\multirow{-2}{2.5cm}{\textit{Strongyloides papillosus}} & oui & 27.14 ± 6.04 \linebreak( 19.09 ; 37.92 ) & \multirow{-2}{*}{\raggedleft\arraybackslash 0.06} & 2.58 ± 0.36 \linebreak( 2 ; 3 ) & \multirow{-2}{*}{\raggedleft\arraybackslash 0.68}\\
		\cmidrule{1-6}
		& non & 30.19 ± 4.60 \linebreak( 17.03 ; 40.15 ) &  & 2.56 ± 0.32 \linebreak( 2 ; 3 ) & \\
		\cmidrule{2-3}
		\cmidrule{5-5}
		\multirow{-2}{2.5cm}{\textit{Nematodirus battus}} & oui & 29.48 ± 4.84 \linebreak( 20 ; 37.13 ) & \multirow{-2}{*}{\raggedleft\arraybackslash 0.42} & 2.44 ± 0.33 \linebreak( 2 ; 3.5 ) & \multirow{-2}{*}{\raggedleft\arraybackslash 0.04*}\\
		\cmidrule{1-6}
		\pagebreak
		\midrule
		& non & 30.19 ± 4.55 \linebreak( 17.03 ; 40.15 ) &  & 2.55 ± 0.32 \linebreak( 2 ; 3.5 ) & \\
		\cmidrule{2-3}
		\cmidrule{5-5}
		\multirow{-2}{2.5cm}{\textit{Moniezia benedeni}} & oui & 26.08 ± 8.01 \linebreak( 19.09 ; 35.86 ) & \multirow{-2}{*}{\raggedleft\arraybackslash 0.26} & 2.50 ± 0.00 \linebreak( 2.5 ; 2.5 ) & \multirow{-2}{*}{\raggedleft\arraybackslash 0.70}\\
		\cmidrule{1-6}
		& non & 30.14 ± 4.62 \linebreak( 17.03 ; 40.15 ) &  & 2.55 ± 0.32 \linebreak( 2 ; 3.5 ) & \\
		\cmidrule{2-3}
		\cmidrule{5-5}
		\multirow{-2}{2.5cm}{\textit{Trichuris} spp.} & oui & 28.15 ± NA \linebreak( 28.15 ; 28.15 ) & \multirow{-2}{*}{\raggedleft\arraybackslash 0.58} & 2.50 ± NA \linebreak( 2.5 ; 2.5 ) & \multirow{-2}{*}{\raggedleft\arraybackslash 0.85}\\
		\cmidrule{1-6}
		& non & 30.04 ± 4.60 \linebreak( 19.09 ; 40.15 ) &  & 2.54 ± 0.31 \linebreak( 2 ; 3 ) & \\
		\cmidrule{2-3}
		\cmidrule{5-5}
		\multirow{-2}{2.5cm}{\textit{Fasciola gigantica}} & oui & 30.27 ± 4.66 \linebreak( 17.03 ; 38.03 ) & \multirow{-2}{*}{\raggedleft\arraybackslash 0.65} & 2.57 ± 0.33 \linebreak( 2 ; 3.5 ) & \multirow{-2}{*}{\raggedleft\arraybackslash 0.54}\\
		\cmidrule{1-6}
		& non & 30.18 ± 4.64 \linebreak( 17.03 ; 40.15 ) &  & 2.54 ± 0.32 \linebreak( 2 ; 3.5 ) & \\
		\cmidrule{2-3}
		\cmidrule{5-5}
		\multirow{-2}{2.5cm}{\textit{Paramphistomum cervi}} & oui & 28.72 ± 3.88 \linebreak( 22.11 ; 35.27 ) & \multirow{-2}{*}{\raggedleft\arraybackslash 0.24} & 2.86 ± 0.23 \linebreak( 2.5 ; 3 ) & \multirow{-2}{*}{\raggedleft\arraybackslash <0,001 ***}\\
		\midrule
		\multicolumn{6}{l}{\textbf{ET}: Ecart-type; \textbf{min}: valeur minimale; \textbf{max}: valeur maximale} \\
		\multicolumn{6}{l}{\textbf{p-values}: P-values obtenues après réalisation du test de Wilcoxon} \\
		\bottomrule
\end{longtable}

\par Il y a un effet statistiquement significatif des espèces \textit{Ostertagia ostertagi} (p = 0,03), \textit{Paramphistomum cervi} (p < 0,001) et \textit{Nematodirus battus} (p = 0,04) sur la NEC. Seule l'espèce \textit{Cooperia} spp. (p = 0,02) a un effet significatif sur l’hématocrite. Les Hématocrites et les NECs moyens des animaux parasités sont globalement inférieurs à ceux des animaux non parasités pour toutes les espèces à l'exception des Trématodes et de l'espèce \textit{Strongyloides papillosus}.

\subsubsection{Effets des intensités d’infestation}

\par La NEC ne montre pas de différence significative entre les intensités d’infestation (tableau \ref{tab:effet_individuel_magnitude}). Cependant le test de Wilcoxon appliqué aux groupes d'animaux lourdement infestés et légèrement infestés par \textit{Ostertagia ostertagi} montre une différence significative au niveau des cas d’anémie.  Les hématocrites moyens et les taux d'animaux anémiés pour l'espèce \textit{Cooperia} spp. en fonction de l'intensité d'infestation sont significativement différents, mais il faudrait davantage de données avant de généraliser ce résultat car il n'y a qu'un seul animal lourdement infesté à cette espèce.
 
\begin{landscape}
 	\begingroup\fontsize{10}{12}\selectfont
\begin{longtable}[!h]{ll>{\centering}p{3cm}cc>{\centering}p{2.5cm}c>{\centering}p{3cm}c}
	
	\caption{\label{tab:effet_individuel_magnitude}Effets des magnitudes d'infestation sur l'hematocrite, la NEC et les proportions d'animaux anémiés} \\
			\toprule
			\textbf{Espèces} & \textbf{Niveaux} & \textbf{PCV moyens ± ET(min; max)} & \textbf{P-values}\textsuperscript{1} & \textbf{n} & \textbf{Cas d'anémies}\linebreak nombre(\%) & \textbf{P-values}\textsuperscript{2} & \textbf{NEC moyennes ± ET(min; max)} & \textbf{P-values}\textsuperscript{1}\\
			\midrule
			\multirow{6}{2cm}{\textbf{\textit{H. contortus}}} &
			\textbf{léger} & 29.88 ± 4.78\linebreak(20.41; 38.46) &  & 84 & 11\linebreak(13,10\%) &  & 2.61 ± 0.30\linebreak(2; 3) & \\
			\cmidrule{2-3}
			\cmidrule{5-6}
			\cmidrule{8-8}
			& \textbf{modéré} & 30.24 ± 4.42\linebreak(20.41; 37.33) &  & 35 & 5\linebreak(14,29\%) &  & 2.46 ± 0.35\linebreak(2; 3) & \\
			\cmidrule{2-3}
			\cmidrule{5-6}
			\cmidrule{8-8}
			& \textbf{lourde} & 29.73 ± 6.70\linebreak(19.09; 38.64) & \multirow{-3}{*}{0.896} & 15 & 3\linebreak(20,00\%) & \multirow{-3}{*}{0.557} & 2.47 ± 0.23\linebreak(2; 3) & \multirow{-3}{*}{0.081}\\
			\midrule
			\multirow{4}{2cm}{\textbf{\textit{O. ostertagi}}} &
			\textbf{léger} & 29.27 ± 4.10\linebreak(22.74; 35.6) &  & 18 & 2(11,11\%) &  & 2.42 ± 0.35\linebreak(2; 3) & \\
			\cmidrule{2-3}
			\cmidrule{5-6}
			\cmidrule{8-8}
			& \textbf{lourde} & 24.49 ± 7.24\linebreak(19.09; 37) & \multirow{-2}{*}{0.064} & 5 & 4(80,00\%) & \multirow{-2}{*}{< 0.001} & 2.40 ± 0.22\linebreak(2; 2.5) & \multirow{-2}{*}{0.105}\\
			\midrule
			\multirow{4}{2cm}{\textbf{\textit{Cooperia} spp.}} &
			\textbf{léger} & 27.83 ± 5.28\linebreak(19.09; 36.88) &  & 17 & 5(29,41\%) &  & 2.44 ± 0.30\linebreak(2; 3) & \\
			\cmidrule{2-3}
			\cmidrule{5-6}
			\cmidrule{8-8}
			& \textbf{modéré} & 20.00 ± NA\linebreak( 20 ; 20 ) & \multirow{-2}{*}{0.038} & 1 & 1(100.00\%) & \multirow{-2}{*}{0.001} & 2.50 ± NA\linebreak(2.5; 2.5) & \multirow{-2}{*}{0.339}\\
			\midrule
			\multirow{4}{2cm}{\textbf{\textit{T. axei}}} &
			\textbf{modéré} & 29.14 ± 4.88\linebreak(19.09; 38.64) &  & 40 & 7(17,50\%) &  & 2.54 ± 0.35\linebreak(2; 3) & \\
			\cmidrule{2-3}
			\cmidrule{5-6}
			\cmidrule{8-8}
			& \textbf{léger} & 29.31 ± 4.45\linebreak(22.5; 37.84) & \multirow{-2}{*}{0.114} & 39 & 5(12,82\%) & \multirow{-2}{*}{0.332} & 2.58 ± 0.35\linebreak(2; 3.5) & \multirow{-2}{*}{0.843}\\
			\midrule
			\multicolumn{9}{l}{\textbf{n}: nombre d'animaux atteint par un niveau d'intensité d'infestation; \textbf{Hem.}: Hématocrite; \textbf{ET}: écart-type; \textbf{min}: valeur minimale; \textbf{max}: valeur maximale} \\
			\multicolumn{9}{l}{\textsuperscript{1}: P-values obtenues à l'aide du test de Kruskal-Wallis} \\
			\multicolumn{9}{l}{\textsuperscript{2}: P-values obtenues à l'aide du test de $\chi$\textsuperscript{2} d'homogenéité des proportions} \\
			\bottomrule
\end{longtable}
\endgroup{}
\end{landscape}

\subsection{Effets sanitaires des co-infestations}
\subsubsection{Effets des combinaisons de parasites sur l'hématocrite}

\begin{table}[!ht]
	\centering
	\caption{Coefficients de la régression linéaire de l'hématocrite par rapport à la présence des combinaisons de parasites}
	\label{tab:model2}
	\begin{tabular}{lcccc}
		\toprule
		\multicolumn{5}{c}{\textbf{Modèle 2}} \\
		\midrule
		& \textbf{Estimation} & \textbf{Erreur Std} & \textbf{Valeur Z} & \textbf{Pr($>|z|$)} \\
		\midrule
		 \textbf{(Intercept)} & 30,3859 & 0,2882 & 105,44 & < 0,001 *** \\
		 combi\_1 & 1,4283 & 2,6055 & 0,55 & 0,5840 \\
		 combi\_2 & -10,3479 & 3,8146 & -2,71 & 0,0071 **\\
		 combi\_3 & -0,9514 & 1,4451 & -0,66 & 0,5108 \\
		 combi\_4 & 2,9118 & 3,0914 & 0,94 & 0,3470 \\
		 combi\_6 & -1,9272 & 3,3890 & -0,57 & 0,5700 \\
		 combi\_7 & -6,7746 & 2,2825 & -2,97 & 0,0033 ** \\
		 combi\_8 & 4,9801 & 3,5111 & 1,42 & 0,1572 \\
		 combi\_9 & 7,7140 & 4,0895 & 1,89 & 0,0603 . \\
		 combi\_10 & 1,9702 & 2,3296 & 0,85 & 0,3984 \\
		 combi\_11 & -0,5249 & 4,1885 & -0,13 & 0,9004 \\
		 combi\_13 & -0,0107 & 1,1987 & -0,01 & 0,9929 \\
		 combi\_14 & 0,0930 & 1,0636 & 0,09 & 0,9304 \\
		 combi\_15 & -3,8674 & 2,7848 & -1,39 & 0,1660 \\
		\midrule
		& \multicolumn{4}{l}{Erreur standard résiduelle: 4,5 sur 286 degrés de liberté}\\
		& \multicolumn{4}{l}{R$^{2}$ multiple: 0,0962 ; R$^{2}$ ajusté: 0,0551}\\
		& \multicolumn{4}{l}{p-value: 0,006 **}\\
		\midrule
		\multicolumn{5}{l}{\textit{Codes de significance : 0 '***' 0,001 '**' 0,01 '*' 0,05 '.' 0,1 ' ' 1}} \\
		\multicolumn{5}{l}{\textbf{combi\_1}: Association des espèces \textit{Ostertagia ostertagi} et \textit{Cooperia} spp.;} \\
		\multicolumn{5}{l}{\textbf{combi\_2}: \textit{O. ostertagi} \& \textit{S. papillosus}; \textbf{combi\_3}: \textit{O. ostertagi} \& \textit{H. contortus};} \\
		\multicolumn{5}{l}{\textbf{combi\_4}: \textit{O. ostertagi} \& \textit{N. battus}; \textbf{combi\_6}: \textit{Cooperia} spp. \& \textit{S. papillosus};} \\
		\multicolumn{5}{l}{\textbf{combi\_7}: \textit{Cooperia} spp \& \textit{H. contortus}; \textbf{combi\_8}: \textit{Cooperia} spp \& \textit{N. battus};} \\
		\multicolumn{5}{l}{\textbf{combi\_9}: \textit{Cooperia} spp. et \textit{T. axei}; \textbf{combi\_10}: \textit{S. papillosus} \& \textit{H. contortus};} \\
		\multicolumn{5}{l}{\textbf{combi\_11}: \textit{S. papillosus} \& \textit{N. battus};
		\textbf{combi\_13}: \textit{H. contortus} \& \textit{N. battus};} \\
		\multicolumn{5}{l}{\textbf{combi\_14}: \textit{H. contortus} \& \textit{T. axei}; \textbf{combi\_15}: \textit{N. battus} \& \textit{T. axei}} \\
		\bottomrule
	\end{tabular}	
\end{table}

Le tableau \ref{tab:model2} résume les coefficients obtenus à l'issue de l'entrainement du modèle de régression linéaire constitué de l'hématocrite comme variable réponse et des associations d'espèces pouvant affecter l'hématocrite comme variables explicatives.
Seules deux combinaisons d'espèces ont présenté des effets statistiquement significatifs sur l'hématocrite. Il s'agit des combinaisons 2 et 7. La combinaison 2 est composée des parasites \textit{Ostertagia ostertagi} et \textit{Strongyloides papillosus} tandis que la combinaison 7 est composée des parasites \textit{Cooperia} spp. et \textit{Haemonchus contortus}.

\subsubsection{Effets des combinaisons de parasites sur la note d'état corporel}

Le modèle adopté pour expliquer les NEC observées est du type régression logistique ordinale. Le tableau \ref{tab:model3} résume les coefficients et les ratios de cotes obtenus à l'issue de l'entrainement du modèle.

\begin{table}[ht]
	\centering
	\caption{Coefficients du modèle de régression logistique ordinale de la note d'état corporel (NEC) en fonction de la présence de combinaisons de parasites
		\label{tab:model3}}
	\begin{tabular}{lccccc}
		\toprule
		\multicolumn{6}{c}{\textbf{Modèle 3}} \\
		\midrule
		& \multicolumn{5}{c}{\textbf{Coefficients}} \\
		\midrule
		& \textbf{Estimation} & \textbf{Erreur Std} & \textbf{Valeur t} & \textbf{Pr(>|t|)} & \textbf{OR}\\
		\midrule
		combi\_44 & 1,17 & -0,47 & 2,50 & 0,01 * & 3,22\\
		combi\_45 & -0,62 & -0,31 & -2,04 & 0,04 * & 0,54\\
		combi\_46 & -0,35 & -0,49 & -0,72 & 0,47 & 0,70\\
		combi\_47 & -0,67 & -0,86 & -0,78 & 0,44 & 0,51\\
		combi\_48 & -0,03 & -1,14 & 2,91 & 0,98 & 0,97\\
		combi\_52 & 2,83 & -0,97 & 2,88 & 0,004 ** & 16,97\\
		\midrule
		& \multicolumn{5}{l}{\textbf{Intercepts}}  \\
		2|2.5 & -1.76 & 0.19 & -9.59 & < 0,001 *** & -\\
		2.5|3 & 1,17 & 0,16 & 7,43 & < 0,001 *** & -\\
		3|3.5 & 5,81 & 1,01 & 5,77 & < 0,001 *** & -\\
		\midrule
		& \multicolumn{5}{l}{Déviance résiduelle: 556,89}\\
		& \multicolumn{5}{l}{AIC: 580,89}\\
		\midrule
		\multicolumn{6}{l}{\textit{Codes de significance : 0 '***' 0,001 '**' 0,01 '*' 0,05 '.' 0,1 ' ' 1}} \\
		\multicolumn{6}{l}{\textbf{combi\_44}: \textit{H. contortus} \& \textit{N. battus}; \textbf{combi\_45} \& \textit{H. contortus} \& \textit{F. gigantica}}\\
		\multicolumn{6}{l}{\textbf{combi\_46}: \textit{H. contortus} \& \textit{T. axei}; \textbf{combi\_47}: \textit{N. battus} \& \textit{F. gigantica}} \\
		\multicolumn{6}{l}{\textbf{combi\_48}: \textit{N. battus} \& \textit{T. axei}; \textbf{combi\_52}: \textit{H. contortus} \& \textit{F. gigantica} \& \textit{T. axei}.}\\
		\bottomrule
	\end{tabular}
\end{table}

La NEC des animaux est affectée significativement par les combinaisons de parasites 44 (OR = 3,22; p-value = 0,01), 45 (OR = 0,54; p-value = 0,04) et 52 (OR = 16,97; p-value = 0,004). La combinaison 44 des parasites est composée de : \textit{Haemonchus contortus} et \textit{Nematodirus battus}, la combinaison 45 de: \textit{Haemonchus contortus} et \textit{Fasciola gigantica} et la combinaison 52 des parasites: \textit{Haemonchus contortus}, \textit{Fasciola gigantica} et \textit{Trichostrongylus axei}. Cette dernière combinaison est celle avec l'effet négatif le plus marqué sur la NEC car la cote d'avoir une NEC inférieure ou égale à 2 chez les animaux atteints est 17 fois celle des animaux indemnes de cette combinaison de parasites. Elle est immédiatement suivie par la combinaison 44 de parasites dont les animaux la présentant ont une cote 3,22 fois celle des animaux non atteints. 

\newpage

\section{Discussion}

Des 10 parasites détectés, \textit{Haemonchus contortus} était le plus fréquent (44,67 \%) suivi de \textit{Fasciola gigantica} (38,33\%), \textit{Trichostrongylus axei} (13,33\%), \textit{Nematodirus battus} (8,33\%), \textit{Ostertagia ostertagi} (7,67\%), \textit{Cooperia} spp. (6,00\%), \textit{Strongyloides papillosus} (4,00\%), \textit{Paramphistomum cervi} (3,67\%), \textit{Moniezia benedeni} (1,33\%), et \textit{Trichuris} spp. (0,33\%). La prévalence obtenue pour \textit{Fasciola gigantica} était similaire à celle trouvée par \shortciteA{RN25} à l’abattoir municipal de Bangangté (33\%), mais largement supérieure aux résultats de \shortciteA{RN27} et \shortciteA{RN28} à Douala et à Yaoundé respectivement. Les découvertes dans ces deux dernières villes appartenant à des aires agroécologiques différentes de notre zone d’étude traduisent une variation de la distribution spatiale des douves au Cameroun. La réserve que l’on pourrait émettre est le fait que ces études aient été conduites dans des abattoirs et donc la provenance des animaux n’est pas toujours locale. Les résultats des travaux, menés par \shortciteA{RN23} dans la Région du Nord-Ouest Cameroun, sont inférieurs aux nôtres : \textit{Trichostrongylus} spp. (5,97\%); \textit{Oesophagostomum} spp. (5,47\%); \textit{Haemonchus} spp. (2.48\%); \textit{Bunostomum} spp. (1,74); \textit{Cooperia} spp. (1,49\%). \textit{Toxocara} spp. (0,24\%); \textit{Ostertagia} spp. (0,50\%); \textit{Nematodirus} spp. (0,74\%); \textit{Trichuris} spp. (0,50\%); \textit{Moniezia} spp (0,50\%); \textit{Eimeria} spp. (0,50\%). Seule la rareté des espèces \textit{Toxocara} spp., \textit{Moniezia} spp. et \textit{Trichuris} spp. est une constante entre les deux zones d’études. Ces faibles taux d’infestation pourraient être liés aux techniques traditionnelles de lutte employées dans le Nord-Ouest. En effet \shortciteA{RN23} ont rapporté l’usage intensif de plantes médicinales en plus des produits anthelminthiques commerciaux afin d’endiguer le parasitisme. Les bergers rencontrés à Bangangté ont admis n’avoir recours qu’aux médicaments pharmaceutiques. Le phénomène de résistance aux anthelminthiques, mais aussi des mauvaises pratiques de management des troupeaux pourraient justifier ces écarts. Le recensement des modes d'élevage des exploitations permettrait d’identifier les pratiques qui influencent les prévalences.

\par Les proportions d’animaux par niveaux de richesse parasitaire variaient de façon significative entre les sites de prélèvements. Cela pourrait être le résultat de l’isolation des habitats qui ne sont pas colonisés de la même façon par les parasites. Leurs habitats préférentiels forment des iles dans lesquelles pénètrent les bovins. Selon \shortciteA{RN45} la densité des populations d’hôtes détermine également l’intensité des transmissions. Or cette densité change peu avec le temps à l’échelle d’un troupeau (à cause de naissances et des mortalités dont le rythme est lent). Elle pourrait varier de façon significative au sein d’une aire géographique donnée avec le chevauchement des zones de pâturages pratiqués par différentes exploitations. Les veaux et les velles étaient davantage victimes de richesses parasitaires lourdes par rapport aux sujets plus âgés. La protection transmise aux veaux via le colostrum est censée les prémunir des helminthes lors de leurs premières expositions en attendant que se mette en place chez eux une immunité acquise selon \shortciteA{RN39}. D’ailleurs, les cas de richesse parasitaire fortes des veaux et velles pourraient être dus au sevrage. Cette phase est délicate, car elle marque le passage de l’immunité maternelle à celle développée par l’animal \shortcite{RN38}. 
L’on a également observé que les femelles étaient plus parasitées que les mâles. Les états physiologiques exprimés par les femelles au cours de leur vie reproductive pourraient expliquer le fait qu’elles soient plus souvent infestées que les mâles chez les bovins de l’Afrique subsaharienne, mais pareille conclusion mérite davantage d’investigations pour confirmation.

\par Les infestations étaient associées à une baisse de l’hématocrite et de la note d’état corporel. L’espèce \textit{O. ostertagi} causait une baisse de l’hématocrite (p = 0,03) conditionnellement à sa charge parasitaire, tandis que \textit{Cooperia} spp. causait une baisse de l’hématocrite (p = 0,02) par sa présence. L’espèce \textit{Cooperia} spp. pourrait alors être virulente, car selon \shortciteA{RN5}, la dangerosité est fonction de la charge parasitaire et de la virulence. L’effet de \textit{Cooperia} spp. sur l’hématocrite suggère qu’en plus de l’érosion de la muqueuse intestinale qu'il provoque, il pourrait y avoir des hémorragies importantes. Les espèces \textit{O. ostertagi} et \textit{N. battus} réduisaient significativement les notes d’états corporels par leur présence chez les animaux. La nocivité de l’espèce \textit{O. ostertagi} a été documentée dans les zones à climat tempéré et peu sous les tropiques. C’est en effet une espèce redoutée par les éleveurs en occident, car elle cause une chute du rendement de carcasse et de la production laitière \shortcite{RN19}. \textit{N. battus} est rarement responsable de l’apparition des signes cliniques. Cette espèce ne devient dangereuse qu’en situation de co-infestation avec d’autres espèces \shortcite{RN16}.

\par Deux combinaisons formaient des associations parasitaires ayant un effet sur l’hématocrite. Il s’agissaient de \textit{Ostertagia Ostertagi} - \textit{Strongyloides papillosus} et de \textit{Cooperia} spp. - \textit{Haemonchus contortus}. Les espèces \textit{Ostertagia ostertagi} et \textit{Haemonchus contortus} sont fortement hématophages selon \shortciteA{RN18} et se localisent dans l’abomasum tandis que \textit{Cooperia} spp. et \textit{Strongyloides papillosus} érodent la muqueuse de l’intestin grêle \shortcite{RN18, RN21}. Dans ces configurations impliquant en amont une perturbation de la fonction gastrique et en aval une baisse de la fonction intestinale, l’animal ne transforme plus efficacement l’herbe qu’il prend et il s’ensuit une émaciation, un déséquilibre des métabolismes énergétiques et protéiques selon \shortciteA{RN20}. L’activité hématophage des parasites \textit{O. ostertagi} et \textit{H. contortus} a dès lors raison de l’hématocrite d’un animal déjà affaibli par des apports nutritifs insuffisants.

\par Deux combinaisons d’espèces parasitaires affectaient la NEC. Il s’agissait d'\textit{Haemonchus contortus} - \textit{Nematodirus battus} et d'\textit{Haemonchus contortus} - \textit{Fasciola gigantica} - \textit{Trichostrongylus axei}. Dans ces combinaisons, \textit{Haemonchus contortus} pourrait être l’élément clé, car selon \shortciteA{RN11} cette espèce supprime certains aspects de la réponse immunitaire intervenant contre les helminthes, favorisant ainsi l’invasion par de nouvelles espèces. On assiste dans ces combinaisons de parasites à l’infestation double de l’estomac et de l’intestin grêle et donc à une perturbation de la fonction digestive. Les espèces avec lesquelles sont associés \textit{H. contortus} dans ces combinaisons à savoir : \textit{N. battus} (responsable de l'atrophie des villosités intestinales), \textit{F. gigantica} (destruction du parenchyme hépatique) et \textit{T. axei} (responsable de l’érosion de la muqueuse intestinale), causent rarement des pathologies à leurs hôtes exceptés lorsque leurs charges sont lourdes ou si elles sont associées à d’autres espèces \shortcite{RN16, RN18, RN21}.

\par Les outils statistiques que nous avons utilisés (modèles statistiques) révèlent l’association existant entre les paramètres sanitaires et certaines combinaisons de parasites sans toutefois indiquer le sens de la causalité. En effet, les combinaisons de parasites pourraient être la cause de la baisse des paramètres sanitaires ou les combinaisons de parasites pourraient être opportunistes de la baisse des paramètres sanitaires. Afin d’établir une relation de cause à effet, il serait nécessaire de contrôler en amont les conditions encadrant l’infestation. Des animaux bien portants, infestés artificiellement et mis en observation, donneraient l’occasion de conclure sur l’impact réel des combinaisons de parasites.
