\chapter{Revue de la littérature}
%\chapter*{Chapitre I: Revue de la littérature}
%\addcontentsline{toc}{chapter}{Chapitre I: Revue de la littérature}

\section{Généralités sur les helminthes parasites du tractus gastro-intestinal des bovins}
\subsection{Position systématique des helminthes gastro-intestinaux des bovins}

Les parasites helminthes gastro-intestinaux des bovins se retrouvent au sein de deux phylums : 
les Nematoda (vers ronds) et les Platyhelminthes (Trématodes et Cestodes). La classe des 
Secernentea regroupeatodes d'importance vétérinaire et contient 
16 superfamilles Trématodes rencontrés chez les bovins appartiennent 
à la sous-classe des Digénéens. Morphologiquement, deux ventouses (l'une orale et l'autre 
ventrale) les caractérisent. Ils sedivisées en individus avec bourses ou sans \shortcite{RN5}. Parmi les 
Cestodes, la présence de quatre acétabulums sur le scolex distingue les membres de l'ordre 
des Cyclophyllidae \shortcite{RN16}. Les  présentent sous deux formes: aplaties dorso-ventralement 
dans leur grande majorité (par exemple \textit{Fasciola} spp.), ou parfois cylindrique 
(comme \textit{Paramphistomum} spp.). La figure \ref{fig:classification-especes}  montre 
la position systématique des genres d'helminthes parasites gastro-intestinaux les plus 
couramment rencontrés à savoir: \textit{Strongyloides}, \textit{Oesophagostomum}, 
\textit{Cooperia}, \textit{Ostertagia}, \textit{Trichos toutes les espèces de Némtrongylus}, \textit{Haemonchus}, 
\textit{Nematodirus}, \textit{Toxocara}, \textit{Trichuris}, \textit{Fasciola}, 
\textit{Paramphistomum} et \textit{Moniezia}.

\begin{figure}
	\centering
	\includegraphics[width = 10cm]{images/Systematique.png}
	\caption[Position systématique des principaux genres d'helminthes parasites 
	gastro-intestinaux des bovins]{\label{fig:classification-especes}Position systématique 
	des principaux genres d'helminthes parasites gastro-intestinaux des bovins. Adapté de
		\shortciteA{RN17} et de \shortciteA{RN5}}
	\label{fig:systematique}
\end{figure}

\subsection{Pathogenèse des helminthes gastro-intestinaux des bovins}

La localisation préférentielle dans le tube digestif et le mode de prise alimentaire 
définissent le pathologenèse des helminthes parasites. La figure 
\ref{fig:diagramme_tractus_digestif} montre leurs localisations préférentielles dans le 
tube digestif.  Les helminthes gastro-intestinaux se nourrissent de sang, des cellules 
du tissus muqueux et sub-muqueux de la lumière intestinale ou du parenchyme d'organes 
annexes. Les espèces hématophages interagissent avec le système hémostatique de l'hôte 
afin d'empêcher la coagulation sanguine lors du repas sanguin. Ainsi elles émettent des 
substances dans le courant sanguin dont la fonction est de détruire le fibrinogène et 
la fibrine ou de se fixer aux facteurs de coagulation \shortcite{RN51}. Les infestations 
concurrentes de l'abomasum et de l'intestin grêle diminuent sensiblement la fonction 
digestive de l'animal. Les hôtes parviennent à compenser la réduction de la motilité 
gastrique et la perturbation des sécrétions acides et de pepsinogène, si l'intestin 
grêle heberge des charges parasitaires insignifiantes \shortcite{RN20}.

\begin{figure}[!ht]
	\centering
	\includegraphics[width = 15cm]{images/Schema-boyaux.png}
	\caption [Diagramme simplifié du tractus gastro-intestinal d'un ruminant montrant les 
	sites anatomiques préférentiels des helminthes]{Diagramme simplifié du tractus 
	gastro-intestinal d'un ruminant montrant les sites anatomiques préférentiels des 
	helminthes}
	\label{fig:diagramme_tractus_digestif}
\end{figure}

\subsubsection{Pathogenèse des Nématodes}

\par Trois espèces colonisent l'estomac vrai du bovin (abomasum ou caillette) : 
\textit{Ostertagia ostertagi}, \textit{Haemonchus contortus} et \textit{Trichostrongylus axei}. 
\par L'espèce \textit{O. ostertagi} est l'une des plus pathogéniques dans les pays à 
climat temperé et occasionne des pertes de productivité importantes \shortcite{RN18, RN19}. 
En effet elle (l'espèce \textit{O. ostertagi}) provoque de l'anorexie, des perturbations 
de la fonction gastro-intestinale et du métabolisme protéique, énergétique et minéral à 
cause de la perturbation de la motilité et de la fonction gastrique \shortcite{RN20}. 
L'ostertagiose de type 1, associée à l'ingestion massive de larves infestantes, apparait 
chez les animaux de moins de 2 ans et cause la diarrhée et l'anorexie. L'ostertagiose 
de type 2, qui survient chez les animaux âgés de 2 à 4 ans voire plus, résulte de 
l'émergence et du développement d'une larve hypobiotique. En plus des signes de 
l'ostertagiose de type 1, le type 2 cause un  \oe dème submandibulaire, de la fièvre 
et de l'anémie \shortcite{RN19}. 
\par La pathologie occasionnée par l'espèce \textit{Haemonchus contortus} s'aggrave 
pendant de lourdes infestations et repose sur l'activité hématophage de cette dernière. 
On observe alors une émaciation progressive de l'animal, une anémie modérée, des 
hémorragies gastriques, des diarrhées et de l'hypoalbuminémie \shortcite{RN21}. 
\par \textit{T. axei} se nourrit en érodant la muqueuse abomasale ou intestinale et 
intoxique ses hôtes à l'aide des déchets métaboliques qu'elle rejette dans la circulation 
sanguine, induisant ainsi quelquefois une déficience thyroïdienne \shortcite{RN18}.

\par Les espèces touchant l'intestin grêle sont plus nombreuses comparativement à 
celles affectant d'autres portions du tube digestif. Dans ce groupe se retrouvent 
les espèces \textit{Cooperia} spp., \textit{Nematodirus} spp., \textit{Toxocara} spp., 
\textit{Strongyloides papillosus} et \textit{Trichostrongylus} spp. Ces helminthes , 
à quelques exceptions près, se nourrissent du tissu muqueux et entrainent une destruction 
sévère des villosités. La capacité de l'intestin à échanger des fluides et des nutriments 
est fortement réduite \shortcite{RN21}. Les espèces \textit{Toxocara} spp. et 
\textit{Strongyloides papillosus} expriment des pathogenèses différentes. En effet,  
\textit{Toxocara} s'accumule au point de former une masse qui obstrue la lumière 
intestinale des veaux, bloquant ainsi le passage d'aliments ingérés de même que 
leur assimilation \shortcite{RN21}. L'espèce \textit{Strongyloides papillosus} 
exhibe une pathogénèse fortement liée à son cycle biologique. En effet, il pénètre 
par effraction cutanée chez son hôte et se déplace à travers le système veineux et 
différents tissus avant de se retrouver dans l'intestin.
Selon \shortciteA{RN18} \textit{S. papillosus} entraine des dommages de 3 ordres :
\begin{itemize}
	\item symptômes cutanés : Démangeaisons et irritation aux sites de pénétration de 
	la larve L3 sur la peau ;
	\item symptômes respiratoires : Lésions du poumon, toux chronique, bronchite, 
	pneumonie, éosinophilie due à la migration larvaire ;
	\item symptômes abdominaux : Diarrhée sanguinolente et contenant du mucus, perte 
	de poids, hydrothorax, ascites. Les sujets lourdement infestés peuvent mourir lors de
	primo-infections.
\end{itemize}

\par \textit{Oesophagostomum radiatum} et \textit{Trichuris} spp. affectent la partie 
distale du tractus digestif à savoir le c\oe cum, le colon et le rectum. Les trichures 
se nourrissent des cellules de la paroi intestinale et de sang. Leurs repas sanguins 
provoquent des pertes de sang négligéables. Mais en de rares occasions, les lésions de 
l'épithélium peuvent entrainer des hémorragies chroniques et de l'anémie \shortcite{RN16}. 
L'espèce \textit{Oesophagostomum radiatum} a reçu le nom de ver nodulaire à cause des 
nodules que forment ses larves en se développant dans la muqueuse de l'intestin grêle et 
du gros intestin \shortcite{RN16}. Ces nodules qui peuvent être palpés à travers le 
rectum affectent la motilité intestinale chez les sujets âgés et occasionnent une 
diarrhée de couleur noire caractéristique \shortcite{RN22}.

\subsubsection{Pathogenèse des Cestodes}
\textit{Moniezia} spp. est d'une importance pathologique si mineure qu'il ne produit 
des effets significatifs qu'en de rares circonstances. Ne possédant pas de tube digestif, 
il se nourrit des nutriments présents dans le bol alimentaire de l'hôte par diffusion 
cutanée et donc peut parfois occasionner des carences en des éléments nutritifs essentiels 
à l'hôte \shortcite{RN18}. 

\subsubsection{Pathogenèse des Trématodes}

\par \textit{Paramphistomum cervi} affecte les préestomacs. Sa forme juvénile cause 
plus de dégâts que sa forme adulte. En effet, cette dernière après l'éclosion de l'\oe uf 
dans le tissu submuqueux du duodénum, migre vers les préestomacs, son habitat définitif. 
Au cours de cette migration, l'apparition d'une entérite et des hémorragies affaiblit 
significativement la condition physique de l'hôte \shortcite{RN16}.
\par Les Trématodes du genre \textit{Fasciola} (encore appelés douves) infestent le 
foie, organe annexe du tube digestif. Les deux espèces les plus importantes sont 
\textit{F. hepatica} que l'on trouve dans les zones tempérées et \textit{F. gigantica} 
qui prédomine dans les régions tropicales plus chaudes. Les dommages causés par les 
douves juvéniles qui pénètrent la paroi intestinale et la capsule de Glisson, sont 
moins pathogéniques que la nécrose que ces derniers induisent dans le parenchyme 
hépatique en s'y déplaçant. Elles se nourrissent d'abord de sang et de cellules hépatiques. 
Une anémie peut survenir à ce stade lors d'infestations massives. Les douves se 
positionnent dans les canaux biliaires dont elles entrainent l'inflammation. Du tissu 
fibreux se forme dans la paroi des canaux biliaires ce qui empêche l'évacuation de la 
bile. La pression rétrograde qui s'ensuit occasionne l'atrophie et la cirrhose du 
parenchyme hépatique. Lors d'infestations massives, la vésicule biliaire est détruite 
et les canaux biliaires érodés laissent passer les douves dans le parenchyme hépatique. 
Ces derniers y provoquent alors de larges abcès \shortcite{RN16}.

\subsection{Situation épidémiologique des helminthes gastro-intestinaux de bovins au Cameroun}
Au Cameroun, les prévalences des helminthes varient en fonction des zones agroécologiques. 
Ainsi \shortciteA{RN23} ont identifié 11 espèces de parasites dans la Région du Nord-Ouest 
avec les prévalences suivantes : \textit{Trichostrongylus} spp. (5.97\%); 
\textit{Oesophagostomum} spp. (5.47\%); \textit{Haemonchus} spp. (2.48\%); \textit{Bunostomum} 
spp. (1.74\%); \textit{Cooperia} spp. (1.49\%). \textit{Toxocara} spp. (0.24\%); 
\textit{Ostertagia} spp. (0.50\%); \textit{Nematodirus} spp. (0.74\%); \textit{Trichuris} 
spp. (0.50\%); \textit{Moniezia} spp. (0.50\%). Dans la même lancée, \shortciteA{RN24} ont 
analysé l'impact sanitaire des co-infestations par les trypanosomes et les parasites 
gastro-intestinaux dans le Département du Mayo-Rey dans la Région du Nord Cameroun. Ils 
ont obtenu une prévalence globale de parasites gastro-intestinaux de 33,62\% avec deux 
types d'\oe ufs d'helminthes décelés : Strongles (96.30\%) et \textit{Toxocara} spp. 
(2.47\%).  \shortciteA{RN25}, dans une étude comparative entre diverses méthodes de 
détection des \oe ufs de douves hépatiques, ont révélé une prévalence de 33\% à l'abattoir 
municipal de Bangangté dans la Région de l'Ouest Cameroun. \shortciteA{RN26} ont 
identifié \textit{Paramphistomum daubneyi}, \textit{Fasciola hepatica} et 
\textit{Dicrocoelium hopes} avec une prévalence globale de 31,15\% dans le département 
de la Vina. Dans une enquête sociologique qui mesurait l'impact financier des saisies 
de foie dégradées par les douves à l'abattoir SODEPA de Yaoundé, \shortciteA{RN28} 
ont déterminé des prévalences de 18\% et 27\% chez les bovins et les moutons respectivement. 
\shortciteA{RN27} dans l'abattoir municipal de Douala ont étudié les facteurs de risque 
de la co-infestation par plusieurs espèces de douves. Ils ont ainsi découvert que 
\textit{Fasciola} spp. infeste seul 12,2\% de bovins, \textit{Dicrocoelium dendriticum} 
25,6\% puis l'association composée de \textit{Fasciola} spp. et de \textit{Dicrocoelium 
dendriticum} infeste 43,6\% d'animaux.

\section{Quelques paramètres de morbidité chez les bovins}
Afin de déterminer l'état sanitaire à l'échelle individuelle ou d'un troupeau, les 
professionnels de l'élevage ont recours à des indicateurs biologiques variés et divers. 
Le contrôle permanent de ces derniers permet non seulement le suivi de la production, 
mais aussi l'évaluation des dégâts causés par d'éventuels agents pathogènes.
Nous nous proposons ici de passer en revue quelques paramètres couramment utilisés.

\subsection{Les indices de production}
La sélection des animaux à l'aide de la masse corporelle précède l'amélioration de la 
production de viande, lors de l'abscence d'informations généalogiques. Ainsi de nombreux 
index biométriques déterminent le potentiel de croissance et de développement musculaire 
des animaux \shortcite{RN29}. L'index de masse, l'index de profondeur du thorax, l'index 
du sternum, le rendement de carcasse et la note d'état corporel, constituent les plus 
importants d'entre eux. La note d'état corporel (NEC) est une méthode d'évaluation de 
la quantité d'énergie métabolisable stockée dans le gras et les muscles. C'est un 
indice corporel couramment utilisé par les zootechniciens, mais aussi par les 
pathologistes \shortcite{RN29}.

\subsection{Les paramètres cliniques}
Les paramètres cliniques renseignent sur les processus pathologiques en cours au sein 
d'un animal.
Le sang et son étude occupent une place privilégiée dans l'investigation des troubles 
sanitaires. La numération de ses éléments figurés peut être interprétée de différentes 
manières selon le groupe de cellules dont le décompte a dévié en dehors des intervalles 
physiologiques normaux. D'ailleurs, la variation de la concentration des leucocytes 
révèle  des infections ou des inflammations \shortcite{RN32}.
L'anémie constitue l'anomalie la plus fréquente et la plus significative de l'hémogramme 
et apparait immédiatement après une perte sanguine, l'hémolyse ou une maladie chronique. 
Les valeurs de référence de l'hématocrite des bovins se situent entre 24 et 48\% 
\shortcite{RN33}. Toute valeur en dessous de 24 est signe d'anémie. Les parasites 
internes et externes causent des pertes sanguines \shortcite{RN32}. En effet, 
\shortciteA{RN37} ont démontré une association significative entre l'indice d'anémie 
FAMACHA\textsuperscript{\circledR} et la présence d'helminthes dans le tube digestif. 
De même \shortciteA{RN35} ont remarqué une baisse significative de l'hématocrite chez 
les animaux co-infectés par la paire trypanosome – strongle. L'infestation par les 
strongles mène à une baisse significative du taux de globules blancs en cas de charge 
parasitaire lourde tandis que les trypanosomes causent de la thrombocytopénie. De même, 
les hémoparasites affectent la concentration des plaquettes sanguines.
\par Les examens à distance et rapproché permettent de récolter des paramètres 
physiologiques dont les déviations des normes standards résultent d'un dysfonctionnement 
de l'organisme, quelquefois causé par un agent pathogène. La fréquence cardiaque, 
la fréquence respiratoire, la température corporelle, la pâleur des muqueuses, la 
consistance des matières fécales (lors de diarrhées ou de constipations), l'aspect 
du pelage (terne ou luisant), la démarche et le comportement de l'animal font partie 
des paramètres cliniques évalués de façon routinière par les vétérinaires \shortcite{RN36}.

\section{La réaction immunitaire des bovins vis-à-vis des helminthes gastro-intestinaux}
Les helminthes, multicellulaires surpassent largement en taille les cellules du système 
immunitaire censées les combattre. Ces dernières ne peuvent les phagocyter en entier. 
La principale stratégie employée par le système immunitaire consiste au fractionnement 
du parasite grace à une attaque massive des cellules T cytotoxiques (qui libèrent leur 
contenu corrosif sur la paroi du parasite), puis phagocyter les éléments obtenus 
\shortcite{RN17}. La seconde stratégie, plus lente à mettre en œuvre, repose sur une 
réponse immunitaire du type 2 (allergique) dont l'initiateur principal, la cellule 
T\textsubscript{H2} (\textit{T-helper cell 2}) stimule la production d'interleukine 
4, 5, 9, 13 et 21. Ces médiateurs chimiques de l'immunité vont à leur tour recruter 
et activer aux sites de fixation des helminthes, les éosinophiles, les basophiles et 
les mastocytes. Ces effecteurs vont à leur tour attaquer la paroi des helminthes et 
amplifier la réaction allergique en sécrétant des cytokines qui vont activer davantage 
de cellules de défense \shortcite{RN17}.
Quelques facteurs intrinsèques aux animaux influencent la bonne tenue du processus de 
neutralisation des parasites.

\subsection{Impact de l'âge des bovins sur leur immunité vis-à-vis des helminthes 
gastro-intestinaux}
Les chances de survie des nouveau-nés dépendent du transfert passif de l'immunité 
entre la mère et le veau grace la consommation du colostrum. Ce transfert doit se 
déroule suffisamment tôt après la naissance, car les veaux ne peuvent absorber les 
gammaglobulines que pendant deux jours \shortcite{RN39}. Le colostrum possède trois 
fonctions : fournir des anticorps maternels au veau (les ruminants n'en reçoivent pas 
pendant la gestation) afin de le protéger des infections et infestations, lui apporter 
des nutriments essentiels et évacuer le méconium de l'intestin par son action 
laxative \shortcite{RN39}. L'immunité propre du veau se développe très lentement 
et devient complète en 1 à 2 saisons de pâturage chez les races exotiques c'est-à-dire 
6 mois environ après le sevrage \shortcite{RN38}. De plus, le nombre d'infestations 
atteint son pic pendant le sevrage et les mois qui suivent, phase délicate pendant 
laquelle le veau acquiert progressivement une immunité propre \shortcite{RN38}.

\subsection{Impact du sexe des bovins sur leur immunité vis-à-vis des helminthes 
gastro-intestinaux}
Lors de la gestation, la régression du corps jaune est bloquée, maintenant les taux 
sanguins de progestérone à des niveaux qui empêchent la destruction du fœtus par le 
système immunitaire maternel \shortcite{RN40}. Cette activité antiinflammatoire de 
la progestérone pourrait favoriser l'installation de parasites. Mais nous n'avons 
pas trouvé d'études qui corroborent cette hypothèse. Par contre, \shortcite{RN41} 
suggèrent que les vaches en lactation et en gestation expriment de lourdes charges 
parasitaires à cause de leur confinement partiel à l'étable et donc leur exposition 
au foin et à de l'eau contaminés par les déjections.

\subsection{La race}
Les races locales ont coévolué avec les parasites endémiques, si bien que le parasitisme 
affecte davantage les races exotiques. Afin de réduire les taux records d'infestation 
observés chez ces dernières, l'une des solutions implémentées a été de les croiser 
avec les races locales afin de corriger héréditairement la productivité et la résistance 
\shortcite{RN43}. Les programmes de sélection exploitent la variation interraciale 
des charges parasitaires afin d'améliorer la résistance aux infestations. De même 
l'identification des gènes responsables de la résistance au sein de races prime dans 
l'amélioration du patrimoine génétique des cheptels bovins \shortcite{RN42}.

\section{Les interactions parasitaires}
\subsection{Les déterminants de la richesse parasitaire}
\subsubsection{La distribution spatiale des hôtes}
La diversité des espèces au sein d'une communauté de parasites dépend de la fragmentation 
et de l'isolation des habitats. En effet l'isolation des hôtes équivaut à des chances 
réduites de dissémination des parasites. Ce n'est pas un hasard qu'aux hôtes fortement 
attroupés correspondent des parasites aux taux de transmission records \shortcite{RN44}. 
Un hôte peut être vu comme une ile, qui offre des conditions de vie adéquates pour un 
parasite. Alors la continuité des habitats des hôtes coïncide avec une distribution 
géographiquement étendue du parasite. Une constante colonisation de nouvelles « iles » 
d'hôtes maintient donc les populations de parasites.
\par À travers la continuité des habitats des hôtes, la transmission des parasites 
s'accélère.
En effet, pour la plupart des infections (et des infestations), la fréquence des 
contacts des hôtes (et donc la transmission) augmente proportionnellement à la densité 
de la population pour une aire géographique donnée. La notion d'« ile » constituée par 
les hôtes n'entrave plus la transmission des parasites en cas de fortes densité des 
hôtes. Au gré des transmissions de diverses espèces parasites, au sein d'une aire 
géographique assez intensément peuplée par une espèce d'hôte, se constitue une 
communauté de parasites \shortcite{RN45}.

\subsubsection{La diversité des hôtes}
La mise en place de communautés de parasites a longuement résulte de l'histoire de 
l'évolution des espèces hôtes à l'échelle des temps géologiques tandis que de nos 
jours elle découle principalement de l'action de l'Homme et se caractérise le plus 
souvent par une perte de biodiversité \shortcite{RN46}. La biodiversité a deux 
répercussions universellement reconnues sur la propagation des parasites : l'amplification 
et la dilution. L'effet d'amplification se définit comme l'augmentation observée de 
la transmission de parasites et des succès de cycles parasitaires quand les espèces 
hôtes sont plus ou moins homogènes et donc peu nombreuses. L'amplification suppose 
une forte spécificité des parasites compétents sur nombre réduit d'espèces hôtes 
\shortcite{RN46}. D'ailleurs \shortciteA{RN47}, lors d'une étude du némabiome de 
grands herbivores, ont démontré que l'espèce de l'hôte explique à 27-53\% la variation 
individuelle de la prévalence parasitaire, la richesse, la composition de la communauté 
de parasites et la diversité phylogénique des parasites. À l'opposé, l'effet de dilution 
correspond à la forte réduction de la transmission des parasites lorsque la biodiversité 
des hôtes est très élevée. La notion de dilution contrairement à celle d'amplification 
suppose une faible compétence des hôtes. La perte des parasites devient donc importante 
chez ces hôtes peu ou pas compétents.

\subsubsection{La distribution des parasites au sein de populations d'hôtes}
La distribution des parasites au sein de populations d'hôtes ou agrégation parasitaire 
constitue un facteur de variation important de la richesse parasitaire d'un individu. 
En effet, la transmission des parasites au sein d'une population d'hôte génère une 
distribution de ces parasites apparemment chaotique. Mais si l'on capture l'image de 
la répartition en un instant, au moyen de l'intensité parasitaire, alors une régularité 
qui singulière s'établit. Pour chaque espèce de parasite, la grande majorité des hôtes 
n'abrite que peu de parasites tandis qu'un nombre réduit d'hôtes en hébergent beaucoup 
\shortcite{RN44}. L'agrégation survient parce que les hôtes varient en susceptibilité 
pour des espèces de parasites données. Cette vulnérabilité elle-même dépend de 
multiples facteurs intrinsèques (génétique, comportement) ou exogènes (environnement) 
à l'hôte.  Un individu qui manifeste une susceptibilité relativement forte provoque 
inévitablement l'accroissement de sa richesse parasitaire \shortcite{RN46}.

\subsection{Les interactions entre infrapopulations de parasites}

\subsubsection{Les interactions bénéfiques pour l'hôte}
La richesse parasitaire n'augmente pas systématiquement le risque de pathologie, 
car de nombreuses espèces causent des dommages négligeables à leur hôte. 
\par En effet, l'apparition de maladie est plutôt corrélée à l'abondance ou la 
prévalence des parasites les plus virulents \shortcite{RN48}. C'est la raison pour 
laquelle pour certains assemblages de parasites l'effet pathogène obtenu s'amoindrit, 
lorsque se constitue une communauté de parasites peu compétents sur communauté très 
diversifiée d'hôtes. Selon \shortciteA{RN49} l'« effet de dilution » pourrait expliquer 
cet état de fait.
\par De plus les espèces parasitaires peuvent se nuire dans le cadre d'une compétition 
interspécifique grâce à la survenue d'une immunité croisée. Ce phénomène aussi appelé 
réaction hétérologue de l'hôte survient lorsque les espèces parasites en présence sont 
similaire du point de leur expression antigénique. Les infestations congénériques sont 
principalement concernés, mais des parasites appartenant à des groupes taxonomiques 
très différents peuvent aussi l'être \shortcite{RN66}. En effet, \shortciteA{RN11} ont 
établi que les immunoglobulines G\textsubscript{1} apparaissant en réponse à une 
invasion par \textit{Trichostrongylus colubriformis} peuvent effectivement s'attaquer 
à l'espèce \textit{Haemonchus contortus} en empêchant le développement larvaire de 
cette dernière. \shortciteA{RN13} ont démontré l'effet négatif de la douve 
\textit{Fasciola hepatica}, sur la morphologie, la quantité et la fertilité des 
kystes d'\textit{Echinococcus granulosus} chez les bovidés domestiques et sauvages. 
Les douves du foie de façon générale présentent une activité immunomodulatrice qui 
influence sensiblement les profils cytokiniques propres à de nombreuses autres 
infections et infestations \shortcite{RN12, RN50}.

\subsubsection{Les interactions néfastes à l'hôte}
L'infestation par certaines espèces d'helminthes favorise l'établissement de nouvelles 
infections au pouvoir pathogène plus accentué. En effet, \shortciteA{RN16} ont montré 
que l'immunomodulation induite par les infestations à \textit{Fasciola} spp. favorise 
l'installation d'une infection à \textit{Mycobacterium tuberculosis}. Les infestations 
helminthiques favorisent la survenue d'une réponse par la mobilisation de lymphocytes 
T\textsubscript{H2} au détriment des lymphocytes T\textsubscript{H1}, ce qui réduit 
la capacité de l'hôte à contrôler les infections aux microparasites intracellulaires 
\shortcite{RN7, RN52}. De même, l'espèce \textit{Haemonchus contortus} supprime 
certains aspects de la réaction immunitaire des moutons, permettant ainsi l'établissement 
de parasites comme \textit{Trichostrongylus colubriformis} \shortcite{RN11}.
\par L'explication de la nocivité d'associations varie d'une étude à une autre. 
En effet pour \shortciteA{RN6} et \shortciteA{RN53} une relation négative existe 
entre la longévité de l'espèce hôte et la richesse parasitaire. En effet, la 
richesse parasitaire et la mortalité qui en découle sont accrues chez les petits 
ruminants comparativement aux bovins domestiques. De plus, \shortciteA{RN6} 
affirment que la saison détermine l'agressivité d'associations parasitaires, 
puisque le maximum de morbidité et d'agrégation parasitaire survient pendant 
l'intersaison. La saison sèche, affecte les élevages semi-extensifs, à cause de 
la rareté des ressources fourragères qui fragilise les animaux.
