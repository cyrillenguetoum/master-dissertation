\chapter{Cadre géographique de l'étude, Matériel et Méthodes}

\section{Cadre géographique de l'étude}
L'Arrondissement de Bangangté situé à 5°09'00” de latitude Nord et 10°31'00” de longitude Est, 
est le chef-lieu du Département du Ndé. Il possède un climat tropical des montagnes avec deux 
grandes saisons : l'une sèche et courte allant de la mi-novembre à la mi-mars et l'autre 
pluvieuse. Les moyennes mensuelles des températures sont comprises entre 14°C et 28°C 
\shortcite{RN54}. Les précipitations annuelles varient entre 1400 et 2500 mm et sont 
inégalement réparties dans les 8 mois pluvieux de l'année \shortcite{RN54}.
\par Les sols de l'Arrondissement sont du type ferrallitique et deviennent progressivement 
volcaniques à la limite avec le Département du Noun. Ils sont alluvionnaires dans les plaines, 
dans les basfonds et latéritiques rougeâtres sur les flancs de quelques sommets 
\shortcite{RN55}.
Le relief est très contrasté et présente trois principales formes : les aires basses localisées 
au Sud, à l'Est et à l'Ouest qui descendent parfois jusqu'à 200 m environ ; les plateaux du Ndé 
où culminent les monts Bangoulap (1542 m), Batchingou (1340 m), Bangoura (1500 m) ; les hautes 
terres dont les altitudes varient de 1400 à 1800 m \shortcite{RN56}.
\par Bangangté se trouve dans le bassin du Noun, arrosé par un réseau hydrographique dense 
constitué de cours d'eau à régimes réguliers ou saisonniers. Les cours sont tortueux du fait 
du relief montagneux et de la profondeur des vallées. Le plus important est le fleuve Noun 
dans lequel se jettent le Kon, le Ngam et le Ndé. La période des crues s'étale le long de la 
saison des pluies, particulièrement entre aout et octobre \shortcite{RN56, RN55}.
\par Les principales formations végétales sont les savanes (arborées, arbustives, herbeuses), 
les poches de forêts primaires, secondaires et de galeries à la lisière des cours d'eau 
\shortcite{RN56}.


\begin{figure}[!ht]
	\centering
	\includegraphics[width = 15cm]{images/carte-master.png}
	\caption[Localisation des fermes échantillonnées]{Localisation des fermes échantillonnées}
	\label{fig:cartes-master}
\end{figure}

\section{Matériel et méthodes}
\subsection{Taille de l'échantillon de bovins}
Les descentes se sont déroulées dans sept exploitations bovines de mai à juillet 2022, en 
début de saison pluvieuse. 
La taille minimale de l'échantillon a été calculée à l'aide de la formule suivante 
\shortcite{RN58} :
\begin{equation*}
	n = \frac{1,96^{2}P_{esp}(1 - P_{esp})}{d^{2}}
\end{equation*}

Avec :
\begin{itemize}
\item n : la taille recherchée de l'échantillon ;
\item $P_{esp}$ : la prévalence espérée. Nous avons considéré un taux de 33\% dans notre 
calcul à partir de l'étude menée par \shortciteA{RN25};
\item d : la précision absolue désirée. Elle vaut 0,05 dans notre cas.
\end{itemize}
	
Après application numérique nous obtenons un effectif de 340 individus. Cette formule est 
utilisée lorsque l'échantillon est tiré suivant le mode aléatoire simple, mais est 
acceptable pour d'autres configurations si la population en dépasse largement la 
taille \shortcite{RN58}. L'effectif total des animaux examinés s'est élevé à 300 individus. 
Chacun d'eux a fait l'objet d'une collecte de fèces, de sang et d'une observation 
visuelle afin d'en obtenir le signalement. 

\subsection{Collecte des fèces et du sang}
L'obtention des fèces a été réalisée en insérant délicatement dans le rectum de l'animal, 
une main gantée soigneusement lubrifiée et enveloppée dans un sac plastique étiqueté au 
préalable. Une fois les matières fécales saisies, la main était retirée et le sac retourné 
puis attaché. Les fèces étaient rapportées au laboratoire à température ambiante.
Nous avons collecté le sang par ponction de la veine jugulaire à l'aide de seringues 
montées sur un support pour tube à EDTA. Les échantillons de sang ont été transportés 
dans une glacière qui contenait de la carboglace congelée afin de les rapprocher de 
l'intervalle de température idéal pour leur conservation [+2 ; +6 °C] comme recommandé 
par \shortciteA{RN59}.

\begin{figure}[!ht]
	\centering
	\begin{minipage}[t]{4cm}
		\includegraphics[width=4cm]{images/IMG-0532.JPG} 
		\caption[Collecte de matières fécales dans le rectum]{ Collecte de matières 
		fécales dans le rectum}
	\end{minipage}
	\begin{minipage}[t]{8cm}
		\includegraphics[width=7cm] {images/IMG-0597.JPG} 
		\caption[Collecte de sang par ponction de la veine de la veine jugulaire]{ 
			Collecte de sang par ponction de la veine de la veine jugulaire}
	\end{minipage}
	
\end{figure}

\subsection{Le signalement des animaux}
Chaque animal a fait l'objet d'une observation visuelle afin d'obtenir son sexe, sa race et
son âge. La présence des tiques sur le pélage a également été notée. 
\par Le sexe s'obtient par une observation de organes génitaux des animaux. En présence 
de testicules et d'un fourreau on conclut à un animal de sexe male tandis qu'en présence 
d'une vulve, on conclut à une femelle.
\par La reconnaissance des caractères morphologiques propres aux races a été la base de leur 
identification chez les animaux. Les métissages interraciaux incontrôlés compliquent davantage
le processus de reconnaissance de la race d'un animal, à tel point qu'il est réduit à 
détermination des caractéristiques raciales dominantes. Nous nous sommes faits accompagnés par
un zootechnicien pour nous aider dans cette tâche.
\par L'âge des animaux a été déterminé par observation de la table dentaire des animaux. Les
détails de cette procédures sont résumés à l'annexe \ref{annex:age-guide}. Les âges ainsi obtenus ont été 
classés en fonction du niveau de leur maturation sexuelle, en trois catégories: veaux et velles
(0 à 1 an), taurillons et génisses (1 à 3 ans) et adultes (plus de 3 ans). La maturation 
sexuelle est considérée comme un indicateur du développement physiologique interne des animaux.


\subsection{Identification des espèces helminthes gastro-intestinaux de bovins}

\subsubsection{Isolation et identification des \oe ufs}
Les matières fécales ont été traitées à l'aide des méthodes de flottaison et de sédimentation.
\par La flottaison a pour but d'extraire les œufs de nématodes et de cestodes des matières 
fécales afin de permettre leur identification et leur décompte. Le principe de la manipulation 
repose sur la différence de densité qui existe entre celle des œufs et celle de la solution 
de flottaison utilisée (une solution saturée d'eau salée). Les œufs, moins denses que la 
solution de flottaison, vont remonter et s'accoler à la paroi supérieure de la lame de 
lecture (du type MacMaster). Le mode opératoire de cette technique consiste à mesurer de 
4 g de fèces transférés dans un mortier de porcelaine puis à ajouter progressivement 56 mL 
d'une solution saline saturée tout en remuant le contenu. Ensuite, le mélange est tamisé 
afin d'éliminer les éléments grossiers. La suspension obtenue est à homogénéisée puis 
introduite dans les chambres de la lame de McMaster à l'aide d'une pipette Pasteur, tout 
en l'inclinant légèrement pour éviter la formation de bulles d'air. Enfin, la lame est 
laissée au repos pendant 5 minutes avant l'observation au microscope optique au grossissement 
x10 \shortcite{RN60}.
\par La technique de sédimentation permet d'isoler et d'identifier les œufs de trématodes. 
Le principe de cette manipulation est basé sur la différence de densité qui existe entre 
la solution de sédimentation (de l'eau simple) et les œufs enfouis dans les matières fécales. 
La solution de sédimentation possède une densité moins élevée que celle des œufs de trématodes 
qui vont précipiter en sa présence. Le mode opératoire consiste à mesurer 3 g de fèces et à 
les transférer dans un mortier en porcelaine. Ensuite, 50 mL d'eau simple est progressivement 
ajouté aux fèces dans le mortier. Le contenu du mortier est écrasé, remué puis tamisé. 
La suspension recueillie est introduite dans un tube et laissée au repos pendant 20 minutes 
afin de précipiter les oeufs au fond du tube. La phase claire
supérieure est soigneusement prélevée grâce à une pipette Pasteur. Une goutte de bleu de 
méthylène est ajoutée aux sédiments. Enfin, une goutte des sédiments est transférée sur 
une lame et recouverte d'une lamelle. L'observation au microscope se fait au grossissement 
x10 puis x40 \shortcite{RN60}. L'identification des \oe ufs grâce à la morphologie et à la 
morphométrie s'est faite en comparaison avec des clichés photographiques rassemblés par 
\shortciteA{RN60} et \shortciteA{RN17}. L'un de ces clichés compilant des photographies d'\oe ufs fréquents chez les bovins est présenté en Annexe \ref{annex:eggs-guide}.

\subsubsection{Calcul des prévalences}

Les données collectées ont été nettoyées et analysées dans les logiciels R \shortcite{RN61} 
version 4.1.2 et R-studio\textsuperscript{\circledR}. La prévalence p des espèces helminthes gastro-intestinaux a été déterminée en divisant le nombre d'animaux infestés par la taille de l'échantillon. L'erreur standard de ce rapport a été calculé grace à la formule 
$ 1,96\sqrt{p(1 - p)/n} $ où p est la prévalence apparente et n la taille de
l'échantillon \shortcite{RN58}. Le calcul de l'erreur standard est basée sur la méthode de 
Wald de calcul de l'intervalle de confiance pour les proportions qui elle-même repose sur une 
approximation normale de la loi binomiale.

\subsubsection{Identification des facteurs de risques à l'infestation des bovins par les 
helminthes gastro-intestinaux}
Le niveau de richesse parasitaire de chaque espèce d'helminthe gastro-intestinal a été calculé
en fonction des caractéristiques de signalement des animaux. Le but recherché a été l'
identification de quelques facteurs de risque à l'infestation par les helminthes gastro-
intestinaux. Les effectifs d'animaux, de même que leurs proportions relatives ont été calculées
pour chaque niveau d'infestation et les proportions ont été comparées à l'aide du test du 
chi-carré d'homogéneité des proportions.

\subsection{Identification des effets sanitaires des infestations par les helminthes 
gastro-intestinaux de bovins}
Les paramètres sanitaires étudiés ont été l'hématocrite et la note d'état corporelle.
L'hématocrite a été déterminée à partir des échantillons de sang et la note d'état corporelle 
a été obtenu par une observation visuelle de l'animal.

\subsubsection{Détermination de l'hématocrite}
L'hématocrite est la proportion des globules rouges dans le sang total. Son obtention 
repose sur la force centrifuge qui s'exerce sur les différentes phases d'un liquide homogène 
lorsqu'il est soumis à une rotation rapide autour d'un axe. La centrifugation produit un 
gradient de phases dont les positions des éléments constitutifs sont fonction de leurs 
densités respectives.  La procédure a débuté par un transfert du sang des tubes de collecte 
à EDTA vers les microtubes capillaires héparinés. Pour chaque microtube, l'une des extrémités 
a été obturée avec de la pâte à sceller (de marque Cristaseal\textsuperscript{\circledR}). 
Les microtubes ont ensuite été centrifugés pendant 10 minutes à une vitesse de 3500 
tours/minute selon les instructions du fabricant (la centrifugeuse utilisée était de 
marque Wirowka\textsuperscript{\circledR}, modèle MPW-310). L'hématocrite a été obtenu 
à l'aide d'un lecteur mécanique gradué.

\subsubsection{Détermination de la note d'état corporel}
La détermination de la note d'état corporel a reposé sur l'attribution à chaque animal de notes 
comprises entre 0 à 5 (0 : cachectique, 1 : trop maigre, 2 : maigre, 3 : bon, 4 : très bon, 5 : 
trop gras) selon le processus décrit par \shortciteA{RN31}. Selon ces derniers, la détermination de la note d'état corporel part de la caractérisation de quelques points anatomiques que sont : la saillie des apophyses transverse et épineuses de la colonne vertébrale, la saillie de l'apophyse iliaque de la hanche, la profondeur du creux de la hanche, la saillie des côtes, la musculature des cuisses, la saillie de la pointe de la hanche et enfin la visibilité du détroit caudal du ligament sacro-tubéral \shortcite{RN31}. Des schémas illustrant la détermination de la note d'état corporel grace à ces structures anatomiques sont présentés à l'annexe \ref{annex:nec-guide}.

\subsubsection{Evaluation des effets qualitatifs des helminthes gastro-intestinaux sur l'hematocrite et la NEC}
Les effets qualitatifs des helminthes gastro-intestinaux ont été évalués par la comparaison, à
l'aide de tests statistiques, des hématocrites moyens et NEC moyennes des animaux infestés par 
diverses espèces d'heminthes gastro-intestinaux et exprimant divers niveaux d'intensité 
parasitaire. Le tableau \ref{tab:guide-magnitude-infestation} montre la classification des 
nombres d'\oe ufs par gramme de fèces de quelques espèces d'helminthes parasites en niveux 
d'intensité parasitaire. Les tests statistiques utilisés ont été le test de Wilcoxon, le test 
de Kruskal-Wallis et le test du Chi-carré d'homogenéité des proportions.

\begin{table}[!h]
	\centering
	\caption[Guide de classification des nombres d'oeufs par gramme de matières fécales 
	en fonction des niveaux d'intensité parasitaire]{Guide de classification des nombres 
	d'oeufs par gramme de matières fécales en fonction des niveaux d'intensité parasitaire 
	chez les bovins selon \shortciteA{RN17}}
	\label{tab:guide-magnitude-infestation} 
	\begin{tabular}[t]{lccc}
		\toprule
		\multirow{2}{3cm}{\bf Espèces} & \multicolumn{3}{c}{\bf Intensité de l'infestation} \\
		\cline{2-4}
		& \textbf{Légère }& \textbf{Modérée} & \textbf{Lourde} \\
		\midrule
		\textbf{\textit{Haemonchus} spp.} & 200 & 200-500 & 500+ \\
		\midrule
		\textit{\textbf{Ostertagia ostertagi}} & 150 & - & 500+ \\
		\midrule
		\textbf{\textit{Trichostrongylus} spp.} & 50 & 50-300 & 500+ \\
		\midrule
		\textbf{\textit{Bunostomum} spp.} & 20 & 20-100 & 100+ \\
		\midrule
		\textbf{\textit{Cooperia} spp.} & 500 & 500-3000 & 3000 \\
		\midrule
	\end{tabular}
\end{table}

\subsection{Identification des effets sanitaires des co-infestations par les helminthes 
gastro-intestinaux de bovins}
\subsubsection{Construction des combinaisons de parasites}
Deux groupes d'espèces ont été constitués à la suite de l'évaluation des effets sanitaires
des infestations par les helminthes gastro-intestinaux. L'un des groupes a contenu les espèces
qui ont montré un effet négatif sur l'hématocrite et l'autre groupe a contenu celles qui ont eu
un effet négatif sur la note d'état corporel.
Des tirages sans remises et non ordonnés des espèces pris dans ces deux groupes ont été 
effectués afin de construire les combinaisons d'espèces. Nous avons supprimé 
des combinaisons qui n'ont pas été retrouvées chez les animaux et réalisé des 
matrices de corrélations afin de réduire le phénomène de "multicollinéarité". Le processus de réduction de la collinéarité entre variables explicatives est davantage expliqué dans l'annexe \ref{annex:statistics-guide}.

\subsubsection{Evaluation quantitative des effets sanitaires des co-infestations dues aux 
helminthes gastro-intestinaux de bovins}

\par Des modèles de régression, linéaire pour l'hématocrite et logistique ordinale pour 
la NEC ont été entrainés afin de mesurer les degrés d'association avec les combinaisons 
d'espèces. D'après \shortciteA{RN67}, l'équation de la régression logistique ordinale 
prend la forme suivante :
\begin{equation*}
	logit(Pr(NEC \leq{2} | X = x))= \beta _{0} + \beta _{1}X_{1} + \beta _{2}X_{2} 
		+ ... + \beta _{n}X_{n}
\end{equation*}
Où :
\begin{itemize}
	\item  $ \beta _{1} , \beta _{2}, ... , \beta _{n} $: sont des coefficients à déterminer.  
	Lorsqu'elles sont transformées à l'aide de la fonction exponentielle, elles deviennent 
	des rapports de cote (\textit{Odds-ratios});
	\item $ \beta _{0} $ une constante représentant le bruit aléatoire des données et les 
	effets non inclus dans le
	modèle ;
	\item $ X_{1} , X_{2}, ..., X_{n} $ , les variables explicatives les plus probantes ;
	\item  $logit(Pr(NEC \leq{2} | X=x))$ est la transformation logistique de la probabilité 
	pour un animal d'avoir une NEC inférieure ou égale à 2, sachant les données $x$ 
	collectées lors de l'étude. 
\end{itemize}