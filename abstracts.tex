\chapter*{Résumé}
\addcontentsline{toc}{chapter}{Résumé}
Les infestations parasitaires réduisent significativement les productions animales avec pour conséquences la hausse de l’insécurité alimentaire et le ralentissement des activités économiques liées à l’élevage. Les effets nocifs des helminthes parasites gastro-intestinaux sont associés à leurs charges, à leurs virulences et aux  combinaisons qu'ils forment. Or peu d’études ont exploré l’influence de ces dernières sur les indices de morbidité des hôtes. Ce travail recherchait la relation entre la présence d'assemblages multispécifiques de parasites et la baisse de deux paramètres sanitaires: l’hématocrite et la note d’état corporel (NEC), de zébus domestiques \textit{Bos indicus}.  Ainsi nous avons réalisé des descentes dans sept exploitations bovines situées dans l’arrondissement de Bangangté, de mai à juillet 2022. En fin de compte, 300 animaux ont fait l’objet de collectes d’échantillons biologiques. L’hématocrite de chaque animal a été déterminé à partir du sang et les espèces de parasites identifiés à travers l'observation des œufs isolés des matières fécales par des méthodes de flottaison et de sédimentation. La NEC a été appréciée grâce à l’examen visuel. Dix espèces d’helminthes ont été identifiées : \textit{Haemonchus contortus} (chez 44,67 \% des animaux), \textit{Fasciola gigantica} (38,33\%), \textit{Trichostrongylus axei} (13,33\%), \textit{Nematodirus battus} (8,33\%), \textit{Ostertagia ostertagi} (7,67 \%), \textit{Cooperia} spp. (6,00\%), \textit{Strongyloïdes papillosus}
(4,00\%), \textit{Paramphistomum cervi} (3,67\%), \textit{Moniezia benedeni} (1,33\%) et \textit{Trichuris} spp. (0,33\%). Les combinaisons de parasites associés à une baisse de l’hématocrite étaient : \textit{Haemonchus contortus} - \textit{Cooperia} spp. et \textit{Ostertagia ostertagi} - \textit{Strongyloides papillosus}, tandis que celles liées à une réduction de la NEC étaient : \textit{Haemonchus contortus} - \textit{Nematodirus battus} (OR = 3,22; p-value = 0,01) et \textit{Haemonchus contortus} - \textit{Fasciola gigantica} - \textit{Trichostrongylus axei} (OR = 16,97; p-value = 0,004). Ces résultats ont mis en lumière une association entre la baisse des paramètres sanitaires et certaines combinaisons de parasites. Mais la relation de causation ne pourra être établie qu'à la suite d’expériences d’infestations artificielles.
\newline
\newline
\textbf{Mots-clés}: parasites, gastro-intestinal, helminthe, co-infestation, hématocrite, NEC.

\chapter*{Abstract}
\addcontentsline{toc}{chapter}{Abstract}
Parasitic infestations significantly reduce livestock production, resulting in increased food insecurity and economic losses. The harmful impacts of gastrointestinal parasitic helminths of the digestive tract are linked with their loads, virulence and multiple species associations. However, few studies have investigated the effect of parasitic associations on cattle morbidity. This work looked for a potential connection between helminthic parasitic species combinations and the decrease of two health parameters, haematocrit and body condition score, in cattle (domestic zebus \textit{Bos indicus}). Thus, we visited cattle farms located in the Bangangté District, from May to July 2022. Ultimately, 300 animals were selected for biological samples collection in seven farms. The haematocrit of each animal was determined from the blood samples and the parasitic species identified through the eggs extracted from feces by flotation and sedimentation methods. The BCS was recorded through careful visual observation of animals. Ten species of helminths were identified: \textit{Haemonchus contortus} (in 44.67\% of animals), \textit{Fasciola gigantica} (38.33\%), \textit{Trichostrongylus axei} (13.33\%), \textit{Nematodirus battus} (8.33\%), \textit{Ostertagia ostertagi} (7.67\%), \textit{Cooperia} spp. (6.00\%), \textit{Strongyloides papillosus} (4.00\%), \textit{Paramphistomum cervi} (3.67\%), \textit{Moniezia benedeni} (1.33\%), and \textit{Trichuris} spp (0.33\%). The combinations of species linked with lowered haematocrit were: \textit{Haemonchus contortus} - \textit{Cooperia} spp. and \textit{Ostertagia ostertagi} - \textit{Strongyloides papillosus}, while those that reduced BCS were: \textit{Haemonchus contortus} - \textit{Nematodirus battus} (OR = 3.22; p-value = 0.01) and \textit{Haemonchus contortus }- \textit{Fasciola gigantica} - \textit{Trichostrongylus axei} (OR = 16.97; p-value = 0.004). These results highlighted a connection between the decrease in health parameters and some combinations of parasites. Experiments involving artificial infestations could permit the establishment of a causal relationship between combination of parasites and health parameters.
\newline
\newline
\textbf{Key words}: Parasites, gastrointestinal, helminths, associations, PCV, BCS.
