\chapter*{Introduction}
\addcontentsline{toc}{chapter}{Introduction}

Au Cameroun, le secteur de l’élevage contribue à 13 \% du produit intérieur brut (PIB). Dominé par le bétail, il constitue une importante source de revenus pour environ 30\% de la population rurale \shortcite{RN1}. En effet, le bétail fournit de la viande, du lait, de la peau et la force pour la traction du trait en agriculture. Les bovins fournissent 35 \% de la viande totale consommée au Cameroun \shortcite{RN1}. Les quantités de viande et de lait tirées du bétail demeurent insuffisantes en raison de la morbidité et la mortalité causée par les pathologies infectieuses \shortcite{RN2}. Cependant l'augmentation de la productivité agricole grace à la résolution des problèmes de l'élevage peut améliorer la sécurité alimentaire, mais aussi générer des emplois et des revenus supplémentaires dans l'économie \shortcite{RN63}.

\par Les infestations parasitaires limitent le rendement des ruminants domestiques. Mais la létalité qu'elles occasionnent ne préoccupe pas les éleveurs la plupart du temps, d'une part en raison de leur caractère subclinique, mais aussi à cause de leur chronicité \shortcite{RN3}. Les parasites helminthes génèrent des pertes économiques et provoquent une baisse de la croissance, de la production laitière et de la fertilité \shortcite{RN4}. La virulence et la charge des helminthes gastro-intestinaux conditionnent les effets nocifs rapportés sur les performances des animaux \shortcite{RN5}. Cependant, les effets pathogènes observés chez ces derniers peuvent également être causés par les actions synergistiques d'espèces membres d'associations particulièrement virulentes \shortcite{RN12, RN13, RN11}. 

\par Dans la pratique, les effets pathogènes ne s’additionnent pas toujours en cas de co-infestation. En effet, les agents parasitaires impliqués peuvent agir indépendamment les uns des autres ou interagir entre eux \shortcite{RN14}. De même, elles peuvent se neutraliser, avantager ou défavoriser l'hôte \shortcite{RN15} avec des conséquences profondes comme la modification de la dynamique des populations de parasites, de leur virulence, de la sévérité des pathologies qu’elles occasionnent et un échec des programmes de contrôle du fait de l’altération de leur épidémiologie \shortcite{RN11}.

\par Il y a donc une urgence de contrôler les parasites helminthes gastro-intestinaux des bovins, dans la perspective d’améliorer la rentabilité des exploitations d’animaux de rente (celle des zébus en particulier). Pour cela l'identification des espèces parasitaires les plus nocives aux animaux est importante, dans un contexte de rarété relative des études établissant l'impact des helminthes gastro-intestinaux sur la santé animale et leurs performances de production. Ainsi ce travail a eu pour objectif général de contribuer à la connaissance de l'impact sanitaire des parasites helminthes gastro-intestinaux de bovins. De façon précise, il s'est agi :
\begin{itemize}
\item d’inventorier, au sein du cheptel bovin de l'Arrondissement de Bangangté, les espèces d’helminthes parasites gastro-intestinaux;
\item d’identifier parmi ces espèces, celles qui sont liées à la baisse de l'hématocrite et de la NEC ;
\item d’identifier au sein de ces espèces, des associations multispécifiques associées à la réduction de l'hématocrite et de la NEC.
\end{itemize}